\clearpage
\subsection{Nilpotent Dynamics}
	Assume $A$ is nilpotent, i.e. without loss of generality we can assume that $A$ is 
	strict lower triangular. From \eqref{eq:HIO:cayley-hamilton} we get 
	\begin{equation}
	c_{k,l} = \delta_{kl} \quad \Rightarrow \quad  \Phi_l = \phi_l 
	\end{equation}
	and by proposition \ref{prop:HIO:phi_injective} each $\Phi$ is injective. 
	

%
%\subsection{Examples}
%Considering the linear system $\mathcal{S}$ we examine some examples.
%
%\begin{example}{Fully Observed System}{}
%	Let $C$ be the $n\times n$ identity matrix and $A$ any $n\times n$ matrix.\\ 
%	Then $\mathcal{I}=\{1,2,\ldots,n\}$ and 
%	$\mathcal{H}_i=\emptyset$ for all $i$. The sufficient condition of theorem 
%	\ref{theorem:HIO:sufficient} reduces to: For $\mu\in\{1,2,\ldots,n\}$ find $(i,l)$ such 
%	that
%	\begin{equation}
%	A^l_{i\mu} \neq 0
%	\end{equation}
%	which is always possible with $i=\mu$ and $l=0$. As expected, a fully observed system 
%	is always HIO. 
%\end{example}
%
%\begin{example}{Single Observable}{}
%	Let $A$ be any $n\times n$ matrix. Interestingly, if we only observe a sum \\$y = 
%	\gamma_1 
%	x_1 + \gamma_2 x_2+\ldots + \gamma_n x_n$, then the system is HIO. If some of 
%	the coefficients are zero, limitations are possible.
%\end{example}
%
%\begin{example}{Simple Cascade}{}
%	If the system is a simple cascade and we observe only the last state, i.e.
%	\begin{equation}
%	A = \begin{pmatrix}
%	0 & 0 & 0 & \hdots \\
%	1 & 0 & 0 & \ddots \\
%	0 & 1 & 0 & \ddots \\
%	0 & 0 & 1 & \ddots \\
%	\vdots & \ddots & \ddots & \ddots
%	\end{pmatrix} \tab{,}
%	C = \begin{pmatrix}
%	0 & 0 & \hdots & 0 & 1
%	\end{pmatrix}
%	\end{equation}
%	then the system is HIO. Obviously, if $A$ is transposed, i.e. the cascade reads 
%	$x_n\to x_{n-1}\to\hdots\to x_1$, it is sufficient to observe $y=x_1$.
%\end{example}
