\subsection{Sufficient Condition}
In most cases $D=\mathbb{1}$ is an appropriate choice, thus $m=n$ and the $\mu$-th column
of $CA^k$ can be written as
\begin{equation}
\left(CA^k \right)_\mu = \sum\limits_{\omega = 1}^n A^k_{\omega\mu} C_\omega
\end{equation}
where $A^k_{\omega\mu}$ is the $(\omega\mu)$ component of $A^k$ and $C_\omega$ is the 
$\omega$-th column of $C$.
Now we can write 
\begin{equation}
y(t) = \sum\limits_{\omega=1}^n \varphi_\omega (t) C_\omega  \label{eq:HIO:y_varphi}
\end{equation}
where 
\begin{equation}
\varphi_\omega(t) :=\sum\limits_{l=0}^{n-1}  
\sum\limits_{\mu=1}^{n} 
 \Phi_l\left[A^l_{\omega\mu} w_\mu \right](t) \quad .
\end{equation}

Now choose an index set $\mathcal{I}\subset \{1,2,\ldots,n\}$ such that $\{C_i|i\in
	\mathcal{I}\}$ are linearly 
	independent and for any $H\in \mathcal{I}^c:=\{1,2,\ldots,n\}\setminus \mathcal{I}$
	 there are unique coefficients $
	\Lambda_i^H$ such that $C_H=\sum_{i\in\mathcal{I}} \Lambda^H_i C_i$. 
	Furthermore introduce the index sets $\mathcal{H}_i$ such that $H\in \mathcal{H}_i 
	\Leftrightarrow \Lambda_i^H = 0$ and $\mathcal{H}_i^c:=\mathcal{I}^c\setminus 
	\mathcal{H}_i$. With this, \eqref{eq:HIO:y_varphi} becomes 
\begin{equation}
y(t) = \sum\limits_{i\in\mathcal{I}} \left(\varphi_i(t) + \sum\limits_{H\in 
	\mathcal{H}^c_i}\Lambda^H_i \varphi_H(t) \right) C_i  \quad .
\end{equation}
To get a condition for HIO, let us set $y\equiv 0$ and by equation coefficients 
\begin{equation}
\varphi_i + \sum\limits_{H\in\mathcal{H}_i^c} \varphi_H 
\equiv 0 \quad \forall i\in\mathcal{I} \quad . \label{eq:HIO:varphi=0}
\end{equation}


\begin{proposition}[Without proof] \label{prop:HIO:Phi_injective}
Each operator $\Phi_l$ is injective, i.e. 
\begin{equation}
\Phi_l[w] \equiv 0\quad \quad \Rightarrow  \quad w \equiv 0  
\end{equation}
and $\Phi_0$ is surjective.
Here "$\equiv$" denotes equality to the zero function and $\Phi_l$ operates 
component-wise on $(w_1,w_2,\ldots,w_m)^\text{T}:[0,T]\rightarrow \mathbb{R}^m$.
\end{proposition}

\begin{definition}
Let $\mathcal{L}$ be an index set. A set 
\begin{equation}
\left\{\quad\Phi_l:L^2([0,T])\to L^2([0,T])\quad|\quad l\in\mathcal{L}\quad \right\}
\end{equation}
of linear operators is called 
\textit{injective set}, if for any functions 
$\{v_l \in L^2([0,T])|l\in\mathcal{L}\}$
the implication 
\begin{equation}
\sum\limits_{l\in\mathcal{L}} 
\Phi_l\left[v_l \right] \equiv 0 \quad \Rightarrow \quad  
v_l \equiv 0 \forall l\in \mathcal{L}
\end{equation}
holds.
\end{definition}

\begin{proposition}\label{prop:HIO:injectiveset}
If $\{\Phi_l|l\in\{0,1,\ldots,n-1\}\}$ defined by \eqref{eq:HIO:Phi} is a injective set 
and if the functions $\{w_\mu\}$ are linearly independent, 
then:\\
\begin{equation}
\begin{aligned}
&\text{If for a}\quad\mu\in\{1,2,\ldots,n\}\quad \exists \quad (i,l)\in \mathcal{I}\times 
\{0,1,\ldots,
n-1\} \\ &\text{such that} \quad A^l_{i\mu} + \sum\limits_{H\in\mathcal{H}_i^c} 
\Lambda^H_i A^l_{H\mu} \neq 0 \\ &\text{then}  \quad w_\mu \equiv 0 \quad .
\end{aligned}
\end{equation}
\end{proposition}
\begin{proof}
Starting with \eqref{eq:HIO:varphi=0} we have for all $i\in\mathcal{I}$
\begin{equation}
\sum\limits_{l=0}^{n-1}\sum\limits_{\mu=1}^n \Phi_l\left[
\left(A^l_{i\mu}+\sum\limits_{H\in\mathcal{H}_i^c} \Lambda_i^H A^l_{H\mu} \right) w_\mu 
\right]
\equiv 0
\end{equation}
and by the definition of an injective set, we get for all $i\in\mathcal{I}$ 
\begin{equation}
\sum\limits_{\mu=1}^n \left(A^l_{i\mu} + \sum\limits_{H\in\mathcal{H}_i^c} \Lambda^H_i 
A^l_{H\mu} \right) w_\mu 
\equiv 0
\end{equation}
and since $\{w_\mu\}$ is a linearly independent set we can treat each $\mu$ separately, 
hence each function $w_\mu$ must vanish at all times if
\begin{equation}
A^l_{i\mu} + \sum\limits_{H\in\mathcal{H}_i^c} \Lambda^H_i A^l_{H\mu} \neq 0 \quad .
\end{equation}
Therefore it is sufficient to find one pair $(i,l)\in \mathcal{I}\times 
\{0,1,\ldots,n-1\}$ for which this coefficient is not zero to argue, that $w_\mu$ must 
be zero at all times.
\end{proof}

\begin{theorem}
Let $\{\Phi_l|l\in\{0,1,\ldots,n-1\}\}$ defined by \eqref{eq:HIO:Phi} be an injective 
set and $\{w_\mu|\mu\in\{1,2,\ldots,n\}\}$ linearly independent functions. If
\begin{equation}
\left.\forall \mu\in\mathcal{M}\exists (i,l)\in \mathcal{I}\times 
\{0,1,\ldots,
n-1\} \right|  A^l_{i\mu} + \sum\limits_{H\in\mathcal{H}_i^c} 
\Lambda^H_i A^l_{H\mu} \neq 0 
\end{equation}
then the system is limited HIO by 
$n-|\mathcal{M}|$. If $|\mathcal{M}|=n$ then the system is HIO.
\end{theorem}
\begin{proof}
Using the preceding proposition the proof is trivial.
\end{proof}







