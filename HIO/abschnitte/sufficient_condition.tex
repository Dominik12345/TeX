\clearpage
\subsection{Sufficient Condition}
In most cases $D=\mathbb{1}$ is an appropriate choice, thus $m=n$ and the $\mu$-th column
of $CA^k$ can be written as
\begin{equation}
\left(CA^k \right)_\mu = \sum\limits_{\omega = 1}^n A^k_{\omega\mu} C_\omega
\end{equation}
where $A^k_{\omega\mu}$ is the $(\omega\mu)$ component of $A^k$ and $C_\omega$ is the 
$\omega$-th column of $C$.
Now we can write 
\begin{equation}
y(t) = \sum\limits_{\omega=1}^n \varphi_\omega (t) C_\omega  \label{eq:HIO:y_varphi}
\end{equation}
where 
\begin{equation}
\varphi_\omega(t) :=\sum\limits_{l=0}^{n-1}  
\sum\limits_{\mu=1}^{n} 
 \Phi_l\left[A^l_{\omega\mu} w_\mu \right](t) \quad .
\end{equation}

Now choose an index set $\mathcal{I}\subset \{1,2,\ldots,n\}$ such that $\{C_i|i\in
\mathcal{I}\}$ are linearly 
independent and for any $H\in \mathcal{I}^c:=\{1,2,\ldots,n\}\setminus \mathcal{I}$
there are unique coefficients $
\Lambda_i^H$ such that $C_H=\sum_{i\in\mathcal{I}} \Lambda^H_i C_i$. 
Furthermore introduce the index sets $\mathcal{H}_i$ such that $H\in \mathcal{H}_i 
\Leftrightarrow \Lambda_i^H \neq 0$ and. With this, \eqref{eq:HIO:y_varphi} becomes 
\begin{equation}
y(t) = \sum\limits_{i\in\mathcal{I}} \left(\varphi_i(t) + \sum\limits_{H\in 
	\mathcal{H}_i}\Lambda^H_i \varphi_H(t) \right) C_i  \quad .
\end{equation}
To get a condition for HIO, let us set $y\equiv 0$ and by equating coefficients 
\begin{equation}
\varphi_i + \sum\limits_{H\in\mathcal{H}_i} \varphi_H 
 = \sum\limits_{l=0}^{n-1} \Phi_l \left[
\sum\limits_{\mu=1}^n  \left( 
A^l_{i\mu} + \sum\limits_{H\in\mathcal{H}_i} \Lambda^H_i A^l_{H\mu}
\right) w_\mu
\right]
 \equiv 0 \quad \forall i\in\mathcal{I}  \quad . \label{eq:HIO:varphi=0}
\end{equation}
Let us define
\begin{equation}
\chi_{i\mu}^l := A^l_{i\mu} + \sum\limits_{H\in\mathcal{H}_i} \Lambda^H_i A^l_{H\mu}
\quad . \label{eq:HIO:chi}
\end{equation}



%\begin{proposition}\label{prop:HIO:notinjectiveset}
%	Let $\{w_\mu|\mu\in\{1,2,\ldots,n\}\}$ be a set of linearly independent functions and 
%	$\{\phi_k|k\in\mathbb{N}_0\}$ as 
%	defined in 
%	\eqref{eq:HIO:Phi}.
%	\begin{equation}	
%	\left.\text{If for a } \mu\in\{1,2,\ldots,n\}\exists (i,k)\in \mathcal{I}\times 
%	\mathbb{N}_0 \right|  
%	\sum\limits_{l=0}^{n-1} c_{k,l} \left(	
%	A^l_{i\mu} + \sum\limits_{H\in\mathcal{H}_i} 
%	\Lambda^H_i A^l_{H\mu}\right) \neq 0 
%	\end{equation}
%	then $w_\mu\equiv 0$.
%\end{proposition}
%\begin{proof}
%	Starting with \eqref{eq:HIO:varphi=0} we get for all $i\in\mathcal{I}$
%	\begin{equation}
%	\sum\limits_{k=0}^\infty \phi_k \left[ 
%	\sum\limits_{l=0}^{n-1} c_{k,l} \left(\sum\limits_{\mu=1}^n A^l_{i\mu} w_\mu + 
%	\sum\limits_{H\in\mathcal{H}_i} \Lambda^H_i\sum\limits_{\mu=1}^n A^l_{H\mu} w_\mu 
%	\right)
%	\right] \equiv 0 \quad .
%	\end{equation}
%	Using proposition \ref{prop:phi_injectiveset} and linearly independence of $w_\mu$ 
%	we get 
%	\begin{equation}
%	\sum\limits_{l=0}^{n-1} c_{k,l}\left( A^l_{i\mu}  + \sum\limits_{H\in\mathcal{H}_i} 
%	\Lambda^H_i A^l{H\mu}	
%	\right) w_\mu \equiv 0 \quad \forall (i,k,\mu)\in \mathcal{I}\times\mathbb{N}_0\times 
%	\{1,2,\ldots,n\} 
%	\end{equation}
%	which means, that if for a given $\mu$ we find a tupel 
%	$(i,k)\in\mathcal{I}\times\mathbb{N}_0$ 
%	where the coefficient of $w_\mu$ does not vanish, we know that $w_\mu\equiv 0$.
%\end{proof}



%\begin{proposition}{}{}%[Simplified version of \ref{prop:HIO:notinjectiveset}]
%\label{prop:HIO:injectiveset}
%If $\{\Phi_l|l\in\{0,1,\ldots,n-1\}\}$ defined by \eqref{eq:HIO:Phi} is a injective set 
%and if the functions $\{w_\mu\}$ are linearly independent, 
%then:\\
%\begin{equation}	
%	\left.\text{If for a } \mu\in\{1,2,\ldots,n\}\exists (i,l)\in \mathcal{I}\times 
%	\{0,1,\ldots,n-1\} \right|  
%	A^l_{i\mu} + \sum\limits_{H\in\mathcal{H}_i} 
%	\Lambda^H_i A^l_{H\mu}\neq 0 
%	\end{equation}
%	then $w_\mu\equiv 0$.
%\end{proposition}
%\begin{proof}
%Starting with \eqref{eq:HIO:varphi=0}, we have for all $i\in\mathcal{I}$
%\begin{equation}
%\sum\limits_{l=0}^{n-1}\sum\limits_{\mu=1}^n \Phi_l\left[
%\left(A^l_{i\mu}+\sum\limits_{H\in\mathcal{H}_i} \Lambda_i^H A^l_{H\mu} \right) w_\mu 
%\right]
%\equiv 0
%\end{equation}
%and by the definition of an injective set and using that the $w_\mu$ are linearly 
%independent we can treat each $\mu$ separately, 
%hence each function $w_\mu$ must vanish at all times if
%\begin{equation}
%A^l_{i\mu} + \sum\limits_{H\in\mathcal{H}_i} \Lambda^H_i A^l_{H\mu} \neq 0 \quad .
%\end{equation}
%Therefore it is sufficient to find one pair $(i,l)\in \mathcal{I}\times 
%\{0,1,\ldots,n-1\}$ for which this coefficient is not zero to argue, that $w_\mu$ must 
%be zero at all times.
%\end{proof}


%\begin{theorem}{Sufficient Condition}{TESTTESTTES} \label{theorem:HIO:sufficient}
%Let $\mathcal{S}$ be the linear dynamic system as defined above. \\
%Let $\{w_\mu|\mu\in\{1,2,\ldots,n\}\}$ be a set of linearly independent\\ functions and 
%$\mathcal{M} \subseteq \{1,2,\ldots,n\}$.  
%According to proposition \ref{prop:HIO:injectiveset}: %\ref{prop:HIO:notinjectiveset} and  
%\begin{equation}
%\left.\text{If }\forall \mu\in\mathcal{M}\exists (i,k)\in \mathcal{I}\times 
%\mathbb{N}_0 \right| \sum\limits_{l=0}^{n-1}c_{k,l}\left( A^l_{i\mu} + \sum\limits_{H\in\mathcal{H}_i} 
%\Lambda^H_i A^l_{H\mu}\right) \neq 0 
%\end{equation}
%respectively, if the operators form an injective set
%\begin{equation*}
%\left.\text{If }\forall \mu\in\mathcal{M}\exists (i,l)\in \mathcal{I}\times 
%\{0,1,\ldots,
%n-1\} \right|  A^l_{i\mu} + \sum\limits_{H\in\mathcal{H}_i} 
%\Lambda^H_i A^l_{H\mu} \neq 0 
%\end{equation*}
%then the system is HIO with $n-|\mathcal{M}|$ limitations.
%\end{theorem}
%\begin{proof}
%Due to proposition \ref{prop:HIO:always_injective}, in the case of linear systems we 
%can use proposition \ref{prop:HIO:injectiveset}, which makes the proof trivial.
%\end{proof}

%\begin{corollary}{}{}
%To make an HIO with $m'$ limitations system HIO, we can set $w_\mu\equiv 0$ for all 
%$\mu\notin\mathcal{M}$ or equivalently adjust the matrix $D$ by setting $D_{\mu\mu}=0$. 
%\end{corollary}


\begin{proposition}{}{} \label{prop:HIO:w=0}
	Assume $y\equiv 0$ and $\{w_\mu\}$ are linearly independent. If there is an $i\in
	\mathcal{I}$ such 
	that $\chi_{i\mu}^l$ \eqref{eq:HIO:chi} can be 
	separated $\chi^l_{i\mu}=\delta_{ll'}
	\chi_{i\mu}$ and if $\Phi_{l'}$ is injective, then each $w_\mu$ with $\chi_{i\mu}
	\neq 0$ 
	must be the zero function.
\end{proposition}
\begin{proof}
	Assume there is such an $i$ and $\Phi_{l'}$ is injective. Starting with 
	\eqref{eq:HIO:varphi=0} we get
	\begin{equation}
	\Phi_{l'} \left[ \sum\limits_{\mu=1}^n\chi_{i\mu} w_\mu \right] \equiv 0
	\quad \Rightarrow \quad 
	\chi_{i\mu} w_\mu \equiv 0 \quad \forall \mu\in\{1,2,\ldots,n\}
	\end{equation}
\end{proof}

\begin{theorem}{Sufficient Condition}{}
	If for a system $\mathcal{S}$ the hidden inputs $\{w_\mu \}$ are linearly independent 
	and if there are enough pairs $(\Phi_l,\chi^l_{i\mu})$ 
	such that proposition \ref{prop:HIO:w=0} holds and for each $\mu$ there is an 
	$\chi_{i\mu}\neq 0$, then the system is HIO.
\end{theorem}
\begin{proof}
	By construction, proposition \ref{prop:HIO:w=0} proves the theorem.
\end{proof}