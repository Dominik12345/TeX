\section{Zero Dynamics}
We follow the ideas of \cite{•}. The zero dynamics of a linear system $(A,B,C)$ is a set of 
triplets $(x,u,y):[0,T]\to \mathbb{R}^n\times \mathbb{R}^m \times \mathbb{R}^p$ that solve the 
system equations and $y\equiv 0$. Assuming we have found such an triplet $(x^*,u^*,0)$, due to 
linearity we can perturb any other triplet $(x,u,y)$ that solves the system equations since
$(x+x^*,u+u^*,y)$ again solves the systems equations and produces the same output $y$ as the 
unperturbed system.
\begin{definition}{}{}
	A subset $V$ of $\mathbb{R}^n$ is called $(A,B)$ invariant if
	\begin{equation}
	A V \subseteq V + \text{Im}\, B \quad .
	\end{equation}
	For the largest $(A,B)$ invariant $V$ we write
	\begin{equation}
	\mathfrak{V} = V \cap \kernel{C}\quad .
	\end{equation}
\end{definition}

\begin{proposition}{}{}
	A trajectory $x:[0,T]\to \mathfrak{V}$ that solves the system 
	equations is part of the zero dynamics of the 
	corresponding system.
\end{proposition}
\begin{proof}
	Assume $x(0)\in\mathfrak{V}$. We define recursively
	\begin{align*}
	A x(0) &= \gamma_0 + \lambda_0 \\
	A\gamma_{i} &= \gamma_{i+1} + \lambda_{i+1}
	\end{align*}
	where each $\gamma_i\in\mathfrak{V}$ and each $\lambda_i\in\text{Im}\, B$. By the 
	definition of $\mathfrak{V}$ we know that this is possible.
	We do a short inductive proof to show
	\begin{equation}
	A^k (\gamma_0+\lambda_0) = \gamma_k + \sum\limits_{l=0}^k A^l \lambda_{k-l} \quad .
	\end{equation}		
	\begin{enumerate}
		\item $k=0$ is trivial.
		\item Assume the statement is correct.
		\begin{align*}
		A^{k+1}(\gamma_0+\lambda_0) &= A \,A^k(\gamma_0+\lambda_0)
		= A \left(\gamma_k +\sum\limits_{l=0}^k A^l\lambda_{k-l} \right) 
		\\
		&= \gamma_{k+1} + \lambda_{k+1} + \sum\limits_{l=0}^k A^{l+1}\lambda_{k-l} 
		= \gamma_{k+1} +\lambda_{k+1} + \sum\limits_{l=1}^{k+1} A^{l} 
		\lambda_{k-(l-1)} \\
		&= \gamma_{k+1} + \sum\limits_{l=0}^{k+1} A^l \lambda_{k+1-l}
		\tab{q.e.d.}
		\end{align*}
	\end{enumerate}

	Furthermore we know that
	\begin{equation}
	x(t) = \e^{At}x(0) + \int\limits_{0}^t \e^{A(t-\tau)} B w(\tau) \, \td \tau 
	\quad .
	\end{equation}
	Expanding the exponentials and shifting the index in the $x(0)$-term yields
	\begin{equation}
	x(t) = x(0) + \sum\limits_{k=0}^\infty A^k \left( 
	\frac{t^{k+1}}{(k+1)!} (\gamma_0+\lambda_0) + \int\limits_0^t \frac{(t-\tau)^k}{k!}
	Bw(\tau) \, \td\tau
	\right)
	\end{equation}
	and using the letter statement
	\begin{equation}
	x(t)=x(0) + \sum\limits_{k=0}^\infty 
	\left\{ \frac{t^{k+1}}{(k+1)!} 
	\left( \gamma_k
	+\sum\limits_{l=0}^k A^l\lambda_{k-l}
	\right) 
	+ A^k \int\limits_0^t \frac{(t-\tau)^k}{k!} Bw(\tau) \, \td \tau
	\right\} \quad .
	\end{equation}
	Applying $C$ leads to
	\begin{equation}
	y(t) =C\sum\limits_{k=0}^\infty  \left\{ 
	\frac{t^{k+1}}{(k+1)!}
	\sum\limits_{l=0}^k A^l \lambda_{k-l} + A^k\int\limits_0^t 
	\frac{(t-\tau)^k}{k!} B w(\tau) \, \td \tau
	\right\}
	\end{equation}
	At order $C A^{k'}$ we have
	\begin{equation}
	\sum\limits_{r=0}^\infty \frac{t^{k'+1+r}}{(k'+1+r)!} \lambda_{r} +
	\int\limits_0^t 
	\frac{(t-\tau)^{k'}}{k'!} B w(\tau) \, \td \tau \quad .
	\end{equation}
	Since each $\lambda_r$ is element of $\text{Im}\,B$ we may find a 
	$w$ that suppresses each term in the summation separately. Without 
	proof we assume that this is always possible.\\
	As a special case we consider $\text{Im}\, B \subseteq \kernel{C}$, 
	which makes the problem trivial.
\end{proof}
	
\subsection{Zero Dynamics vs. HIO}
In the following example we see, that HIO and zero dynamics are closely 
related but not equivalent.

\begin{example}{}{}
	Consider a dynamic system with the matrices
	\begin{equation}
	A = \begin{pmatrix}
	0 & 0 & 1 \\ 1 & 0 & 0 \\ 0 & 1 & 0
	\end{pmatrix}		 \tab{,}
	B = \begin{pmatrix}
	1 & 0 \\ 0 & 1 \\ 0 & 0
	\end{pmatrix}
	\tab{and} C =
	\begin{pmatrix}
	1 & 0 & 0 \\ 0 & 1 & 0
	\end{pmatrix} \quad .
	\end{equation}
	For $\mathfrak{V} := \text{span}\{(0,0,1)^T\}$ we find  
	$A\mathfrak{V}=\text{span}\{(1,0,0)^T\}$ is a subset of 
	$\text{Im}\, B$ and thus $\mathfrak{V}$ is $(A,B)$-invariant. \\
	For instance the set
	\begin{equation}
	x^*(0) = \begin{pmatrix}
	0 \\ 0 \\ a
	\end{pmatrix}
	\tab{and}
	w^*(t) = \begin{pmatrix}
	- a \\
	0
	\end{pmatrix}
	\end{equation}
	leads to a constant
	\begin{equation}
	x^*(t) = \begin{pmatrix}
	0 \\ 0 \\ a
	\end{pmatrix}
	\end{equation}
	which produces zero output $y=Cx^*$ with nonvanishing hidden inputs 
	$w^*$. This 
	shows that the zero dynamics of the system is not trivial. \\
	At the same time we find that $CB$ is injective which means the 
	system is HIO and thus zero output should imply zero hidden intput.
\end{example}

As illustrated by the example, HIO does not imply trivial zero dynamics. 
The reason for that is, HIO is independent of the initial value 
$x^*(0)$, whereas the hidden input $w^*$ of the zero dynamics may only 
work for specific initial values $x^*(0)$. If, for instance, in the above 
example $x^*(0)$ would not have zeroes in the first two components, 
it is not possible to find a nontrivial hidden input.

Consider the following example
\begin{example}{}{}
	A dynamic system with the matrices
	\begin{equation}
	A = \begin{pmatrix}
	0 & 0 & 1 \\ 1 & 0 & 0 \\ 0 & 1 & 0
	\end{pmatrix}		 \tab{,}
	B = \begin{pmatrix}
	1 & 0 \\ 0 & 1 \\ 0 & 0
	\end{pmatrix}
	\tab{and} C =
	\begin{pmatrix}
	0 & 1 & 0 \\ 0 & 0 & 1
	\end{pmatrix} \quad .
	\end{equation}
	We find
	\begin{equation}
	\text{Im}\,B = \text{span} \left\{
	\begin{pmatrix} 1 \\ 0 \\ 0 \end{pmatrix} ,
	\begin{pmatrix} 0 \\ 1 \\ 0 \end{pmatrix}
	\right\} \tab{and}
	\kernel{C} = \text{span} \left\{
	\begin{pmatrix} 1 \\ 0 \\ 0 \end{pmatrix}
	\right\}  \quad .
	\end{equation}
	We notice 
	\begin{equation}
	A \kernel{C} \subseteq \text{Im}\, B
	\end{equation}	
	and therefore clearly see that $\kernel{C}$ is $(A,B)$-invariant.\\
	The system is not HIO for $CB$ has rank $1$.
	\end{example}
	As one can see in the above example, as soon as the image of $A$ restricted to the 
	kernel of $C$ is a subset of the image of $B$ is nonempty, the zero dynamics are 
	nontrivial. In the following proposition we show that for not HIO systems this will 
	always be the case.

\begin{proposition}{HIO is necessary for Trivial Zero Dynamics}{}
	For a linear system of $(A,B,C)$ 
	\begin{equation}
	\text{trivial zero dynamics} \quad \Rightarrow \quad \text{HIO} \quad
	\end{equation}	
	or equivalently
	\begin{equation}
	\mathfrak{V} = \{0\} \quad \Rightarrow \quad CA^kB\quad \text{injective}
	\end{equation}
	for the first nonvanishing $CA^kB$.
\end{proposition}
\begin{proof}
	We show that $\neg$HIO leads to nontrivial zero dynamics.\\
	Assume for simplicity that $CB \neq 0$ and $B$ injective. 
	\begin{enumerate}
	\item $B$ cannot be $0$. Thus $\text{Im}\,B\supset \{0\}$ while $\kernel{B}=\{0\}$. 
	\item Since $\kernel{CB}\supset
	\{0\}$ we know that $\text{Im}\, B \cap \kernel{C}\supset \{0\}$. \\
	Therefore $\kernel{C}\supset \{0\}$.
	\item If $A$ has full rank, then $\dim{\kernel{CA} }=\dim{\kernel{C}}>0$.\\
	If $A$ has rank less than $n$, $A$ is not injective hence $CA$ is not 
	injective. \\
	In either case $\kernel{CA}\supset \{0\}$. 
\end{enumerate}		
	If we now choose
	\begin{equation}
	i
	\end{equation}
	
\end{proof}