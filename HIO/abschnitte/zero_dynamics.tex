\section{Zero Dynamics}
We follow the ideas of \cite{BergerIlchmann}. 
The \textit{zero dynamics (ZD)} of a linear system $(A,D,C)$ is a set of 
triplets $(x,w,y):[0,T]\to \mathbb{R}^n\times \mathbb{R}^m \times \mathbb{R}^p$ that solve 
the system equations and produce $y\equiv 0$.\\

Assume we have found such an triplet 
$(x^*,w^*,0)$. Due to linearity we can perturb any other triplet $(x,w,y)$ that solves the 
system equations by a ZD triplet, since
$(x+x^*,w+w^*,y)$ again solves the system equations and produces the same output $y$ as 
the unperturbed system. \\

%\begin{corollary}{}{}
%	Let $(A,B,C,D)$ the matrices of a dynamical system
%	\begin{equation}
%	\dot{x} = Ax + Bu + Dw \tab{,} x(0)=x_0 \tab{,}
%	y = Cx \quad .
%	\end{equation}
%	Let $\Delta x$, $\Delta w$ and $\Delta y$ denote the differences between 
%	two systems with the same known input $u$ and initial value $x_0$, then 
%	\begin{equation}
%	\Delta \dot{x} = A\Delta x + D \Delta w \tab{,} \Delta x(0)=0 \tab{,} \Delta y = 
%	C \Delta x \quad .
%	\end{equation}
%	Obviously		
%	\begin{equation}
%	D \quad \text{not injective} \quad \Rightarrow\quad \text{not HIO and nontrivial 
%	zero dynamics}
%	\end{equation} 
%	
%\end{corollary}

\begin{definition}{}{}
	A subset $V$ of $\mathbb{R}^n$ is called $(A,D)$ invariant if
	\begin{equation}
	A V \subseteq V + \text{Im}\, D \quad .
	\end{equation}
	For the largest $(A,D)$ invariant $V$ we write
	\begin{equation}
	\mathfrak{V} := V \cap \kernel{C}\quad .
	\end{equation}
	We say a system has \textit{trivial zero dynamics}, if the only triplet $(x,w,y)$ with 
	zero output is $(0,0,0)$. 
\end{definition}
%\begin{corollary}{}{}
%	If the matrix $D$ is not injective, then the ZD cannot be trivial. This is 
%	clear because we can always choose $x=0\in\mathfrak{V}$ and $w\in\kernel{D}$ which 
%	produces zero output.\\
%
%	If $\kernel{C}\cap \kernel{A} \neq \{0\}$, then the ZD cannot be trivial. To see this, 
%	choose $V=\kernel{C}\cap \kernel{A}$. This is clearly $(A,D)$-invariant and subset of 
%	$\kernel{C}$.	\\
%	
%	If $D$ is not injective, the system is not HIO and has nontrivial ZD.	
%\end{corollary}
%\begin{proposition}{}{}
%	A trajectory $x:[0,T]\to \mathfrak{V}$ that solves the system 
%	equations is part of the zero dynamics of the 
%	corresponding system, i.e. if we have an initial value $x_0\in \mathfrak{V}$, it is 
%	possible to find an input $u$ such that $x(t)\in\mathfrak{V}$ at all times.
%\end{proposition}
%\begin{proof} We do not give a rigorous proof but only make the proposition plausible. A 
%	more comprehensive treatment can be found in \cite{BergerIlchmann}.\\
%	Assume $x(0)\in\mathfrak{V}$. We define recursively
%	\begin{align*}
%	A x(0) &= \gamma_0 + \lambda_0 \\
%	A\gamma_{i} &= \gamma_{i+1} + \lambda_{i+1}
%	\end{align*}
%	where each $\gamma_i\in\mathfrak{V}$ and each $\lambda_i\in\text{Im}\, B$. By the 
%	definition of $\mathfrak{V}$ we know that this is possible.
%	We do a short inductive proof to show
%	\begin{equation}
%	A^k (\gamma_0+\lambda_0) = \gamma_k + \sum\limits_{l=0}^k A^l \lambda_{k-l} \quad .
%	\end{equation}		
%	\begin{enumerate}
%		\item $k=0$ is trivial.
%		\item Assume the statement is correct.
%		\begin{align*}
%		A^{k+1}(\gamma_0+\lambda_0) &= A \,A^k(\gamma_0+\lambda_0)
%		= A \left(\gamma_k +\sum\limits_{l=0}^k A^l\lambda_{k-l} \right) 
%		\\
%		&= \gamma_{k+1} + \lambda_{k+1} + \sum\limits_{l=0}^k A^{l+1}\lambda_{k-l} 
%		= \gamma_{k+1} +\lambda_{k+1} + \sum\limits_{l=1}^{k+1} A^{l} 
%		\lambda_{k-(l-1)} \\
%		&= \gamma_{k+1} + \sum\limits_{l=0}^{k+1} A^l \lambda_{k+1-l}
%		\tab{q.e.d.}
%		\end{align*}
%	\end{enumerate}
%
%	Furthermore we know that
%	\begin{equation}
%	x(t) = \e^{At}x(0) + \int\limits_{0}^t \e^{A(t-\tau)} B w(\tau) \, \td \tau 
%	\quad .
%	\end{equation}
%	Expanding the exponentials and shifting the index in the $x(0)$-term yields
%	\begin{equation}
%	x(t) = x(0) + \sum\limits_{k=0}^\infty A^k \left( 
%	\frac{t^{k+1}}{(k+1)!} (\gamma_0+\lambda_0) + \int\limits_0^t \frac{(t-\tau)^k}{k!}
%	Bw(\tau) \, \td\tau
%	\right)
%	\end{equation}
%	and using the letter statement
%	\begin{equation}
%	x(t)=x(0) + \sum\limits_{k=0}^\infty 
%	\left\{ \frac{t^{k+1}}{(k+1)!} 
%	\left( \gamma_k
%	+\sum\limits_{l=0}^k A^l\lambda_{k-l}
%	\right) 
%	+ A^k \int\limits_0^t \frac{(t-\tau)^k}{k!} Bw(\tau) \, \td \tau
%	\right\} \quad .
%	\end{equation}
%	Applying $C$ leads to
%	\begin{equation}
%	y(t) =C\sum\limits_{k=0}^\infty  \left\{ 
%	\frac{t^{k+1}}{(k+1)!}
%	\sum\limits_{l=0}^k A^l \lambda_{k-l} + A^k\int\limits_0^t 
%	\frac{(t-\tau)^k}{k!} B w(\tau) \, \td \tau
%	\right\}
%	\end{equation}
%	At order $C A^{k'}$ we have
%	\begin{equation}
%	\sum\limits_{r=0}^\infty \frac{t^{k'+1+r}}{(k'+1+r)!} \lambda_{r} +
%	\int\limits_0^t 
%	\frac{(t-\tau)^{k'}}{k'!} B w(\tau) \, \td \tau \quad .
%	\end{equation}
%	Since each $\lambda_r$ is element of $\text{Im}\,B$ we may find a 
%	$w$ that suppresses each term in the summation separately. Without 
%	proof we assume that this is always possible.\\
%	As a special case we consider $\text{Im}\, B \subseteq \kernel{C}$, 
%	which makes the problem trivial.
%\end{proof}
%	
%\subsection{Zero Dynamics vs. HIO}
%In the following example we see, that HIO and ZD are closely 
%related but not equivalent.
%
%\begin{example}{}{}
%	Consider a dynamic system with the matrices
%	\begin{equation}
%	A = \begin{pmatrix}
%	0 & 0 & 1 \\ 1 & 0 & 0 \\ 0 & 1 & 0
%	\end{pmatrix}		 \tab{,}
%	B = \begin{pmatrix}
%	1 & 0 \\ 0 & 1 \\ 0 & 0
%	\end{pmatrix}
%	\tab{and} C =
%	\begin{pmatrix}
%	1 & 0 & 0 \\ 0 & 1 & 0
%	\end{pmatrix} \quad .
%	\end{equation}
%	For $\mathfrak{V} := \text{span}\{(0,0,1)^T\}$ we find  
%	$A\mathfrak{V}=\text{span}\{(1,0,0)^T\}$ is a subset of 
%	$\text{Im}\, B$ and thus $\mathfrak{V}$ is $(A,B)$-invariant. \\
%	For instance the set
%	\begin{equation}
%	x^*(0) = \begin{pmatrix}
%	0 \\ 0 \\ a
%	\end{pmatrix}
%	\tab{and}
%	w^*(t) = \begin{pmatrix}
%	- a \\
%	0
%	\end{pmatrix}
%	\end{equation}
%	leads to a constant
%	\begin{equation}
%	x^*(t) = \begin{pmatrix}
%	0 \\ 0 \\ a
%	\end{pmatrix}
%	\end{equation}
%	which produces zero output $y=Cx^*$ with nonvanishing hidden inputs 
%	$w^*$. This 
%	shows that the zero dynamics of the system is not trivial. \\
%	At the same time we find that $CB$ is injective which means the 
%	system is HIO and thus zero output should imply zero hidden intput.
%\end{example}
%
%As illustrated by the example, HIO does not imply trivial ZD but it is easy to 
%see that trivial ZD imply HIO.
%%
%%Consider the following example
%%\begin{example}{}{}
%%	A dynamic system with the matrices
%%	\begin{equation}
%%	A = \begin{pmatrix}
%%	0 & 0 & 1 \\ 1 & 0 & 0 \\ 0 & 1 & 0
%%	\end{pmatrix}		 \tab{,}
%%	B = \begin{pmatrix}
%%	1 & 0 \\ 0 & 1 \\ 0 & 0
%%	\end{pmatrix}
%%	\tab{and} C =
%%	\begin{pmatrix}
%%	0 & 1 & 0 \\ 0 & 0 & 1
%%	\end{pmatrix} \quad .
%%	\end{equation}
%%	We find
%%	\begin{equation}
%%	\text{Im}\,B = \text{span} \left\{
%%	\begin{pmatrix} 1 \\ 0 \\ 0 \end{pmatrix} ,
%%	\begin{pmatrix} 0 \\ 1 \\ 0 \end{pmatrix}
%%	\right\} \tab{and}
%%	\kernel{C} = \text{span} \left\{
%%	\begin{pmatrix} 1 \\ 0 \\ 0 \end{pmatrix}
%%	\right\}  \quad .
%%	\end{equation}
%%	We notice 
%%	\begin{equation}
%%	A \kernel{C} \subseteq \text{Im}\, B
%%	\end{equation}	
%%	and therefore clearly see that $\kernel{C}$ is $(A,B)$-invariant.\\
%%	The system is not HIO for $CB$ has rank $1$.
%%	\end{example}
%%	As one can see in the above example, as soon as the image of $A$ restricted to the 
%%	kernel of $C$ is a subset of the image of $B$ is nonempty, the zero dynamics are 
%%	nontrivial. In the following proposition we show that for not HIO systems this will 
%%	always be the case.
%
%

\begin{proposition}{HIO is necessary for Trivial ZD}{}
	For a linear system $(A,C,D)$ 
	\begin{equation}
	\text{trivial ZD} \quad \Rightarrow \quad \text{HIO} \quad .
	\end{equation}
\end{proposition}
\begin{proof}
	We show that $\neg$HIO leads to nontrivial ZD.
	Let $\mathfrak{V}$ be the largest $(A,D)$-invariant subset included in $\kernel{C}$. 
	By definition
	\begin{equation}
	\forall x\in \mathfrak{V} \quad \exists (x_0,w_0) \in \mathfrak{V}\times \mathbb{R}^m 
	\quad | \quad Ax = x_0 + Dw_0 \quad .
	\end{equation}
	Since $x_0$ is again element of $\mathfrak{V}$ we can find $(x_1,w_1)$ solving the 
	analogous equation. Thus if we multiply the latter equation with $CA^n$,
	$n\in\mathbb{N}_0$, and recursively insert $(x_k,w_k)$ and $Cx_k=0$ we get
	\begin{equation}
	CA^{k+1} x = M_k W_k
	\end{equation}
	where $W_k = [w_k,w_{k-1},\ldots,w_0]^\text{T}$. We notice the similarity to the last 
	HIO theorem and define $\Delta_k$ as before. Assume we cannot find an integer $K$ that 
	makes $\Delta_K$ trivial ($\neg$HIO). Then  
	we can find a sequence $(\hat{w}_k)_{k\in\mathbb{N}_0}$ such that each 
	$\hat{W}_k\in\kernel{M_k}$ and therefore the zero dynamics is nontrivial.
\end{proof}