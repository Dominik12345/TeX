\section*{Hidden Input Observability}
Consider a dynamic system $\mathcal{S}$
\begin{align}
\frac{\td x_w}{\td t} &= Ax_w(t) + Bu(t) + Dw(t) \\
y_w(t) &= Cx_w(t) \\
x_w(0) &= x_0
\end{align}
where $x_w$, $u$, $w$ and $y_w$ map $[0,T]$ onto $\mathbb{R}^n$, 
$\mathbb{R}^{m'}$, $\mathbb{R}^m$ and $\mathbb{R}^p$, respectively, and 
$A$, $B$, $C$, $D$ are matrices of suitable dimensions. We assume the function $u$ is known 
and thus called the \textit{known input}, the function $w$ is unknown hence called 
\textit{unknown input}. The closed form solution for $y_w$ is
\begin{equation}
y_w(t) = C \int\limits_0^t \exp(A(t-\tau)) (Bu(\tau) + Dw(\tau) ) \, \td 
\tau \quad .
\end{equation}

\begin{definition}
If for $\mathcal{S}$ the implication
\begin{equation}
y_{w}(t) = y_{\hat{w}}(t) \quad \forall t\in [0,T]
\quad \Rightarrow \quad 
 w = \hat{w} \quad \text{a.e.}
\end{equation}
holds, $\mathcal{S}$ is called \textit{hidden input observable (HIO)}.
\end{definition}

Our aim is to find necessary or sufficient conditions for the hidden input observability 
of linear systems.\\

Due to linearity, $\mathcal{S}$ is HIO if and only if
\begin{equation}
y(t):=C\int\limits_0^t \exp(A(t-\tau))Dw(\tau)\,\td\tau = 0 \quad 
\forall t\in[0,T] \quad \Rightarrow \quad 
w = 0 \quad \text{a.e.} \label{eq:introduction_alt:1}
\end{equation}

\subsection*{Rearranging the equation}
By the Cayley-Hamilton theorem, for any $k\in\mathbb{N}_0$ there are coefficients $c_{k,l}$ 
such that 
\begin{equation}
A^k = \sum\limits_{l=0}^{n-1} c_{k,l} A^l \quad .
\end{equation}
Defying
\begin{equation}
\Phi_l[w](t) :=  \int\limits_0^t \sum\limits_{k=0}^\infty c_{k,l}\frac{(t-\tau)^k}{k!}  
w(\tau) \, \td\tau 
\end{equation}
equation \eqref{eq:introduction_alt:1} can be written as
\begin{equation}
y(t) = \sum\limits_{l=0}^{n-1} CA^lD \, \Phi_l[w](t)
\quad . \label{eq:introduction_alt:operator}
\end{equation}
A system is hidden input observable if and only if the linear operator
\begin{equation}
\sum\limits_{l=0}^{n-1} C A^l D \, \Phi_l
\end{equation}
is injective in the sense that only the $\mathbb{R}^m$ zero function is mapped onto 
the $\mathbb{R}^p$ zero function.

\subsection*{Connection to target controllability}
In the literature on optimal control theory it is convenient to introduce the 
\textit{controllability matrix} \cite{Luenberg},\cite{Barabasi_k-walk}
\begin{equation}
M = [CD,CAD,CA^2D,\ldots,CA^{n-1}D] \in \mathbb{R}^{p\times nm} \quad .
\end{equation}
\begin{definition}
If $\rank{M} = p$, then $\mathcal{S}$ is called \textit{target controllable}. 
Obviously $p\leq nm$ is crucial.
\end{definition}
\begin{proposition}[Proof in \cite{Luenberg}]
If $\mathcal{S}$ is target controllable, then the operator 
\eqref{eq:introduction_alt:operator} is surjective.
\end{proposition}
We can rewrite \eqref{eq:introduction_alt:operator} as
\begin{equation}
y(t) = M \begin{bmatrix}
\Phi_0[w](t) \\ \Phi_1[w](t) \\ \vdots \\ \Phi_{n-1}[w](t)
\end{bmatrix}
\end{equation}
and by the dimension formula we get
\begin{equation}
\rank{M} + \dim{\kernel{M}} = nm
\end{equation}
and if $\mathcal{S}$ is target controllable
\begin{equation}
\dim{\kernel{M}} = nm - p \geq 0 \quad .
\end{equation}

\subsubsection*{Example}
Consider the case $p=n=m > 1$ and $C=D=\mathbb{1}$ the identity matrix in 
$\mathbb{R}^{n\times n}$. Then $\rank{M}=n$ so that the system is target controllable and 
$\dim{\kernel{nm-p}} > 0$. If we take the zero matrix for $A$, we get coefficients 
\begin{equation}
c_{k,l} = \left\{
\begin{aligned}
 &1 \quad\text{if} \quad k=l=0 \\ &0 \quad \text{else}
\end{aligned}
\right.
\end{equation}
and
\begin{equation}
y(t) = \int\limits_0^t w(\tau) \, \td \tau
\end{equation}
which is a Volterra-equation of the first kind and thus injective.\\

If we closer investigate $M$ we see that the set of
\begin{equation}
\begin{bmatrix}
0 \\ v_1 \\v_2 \\ \vdots \\ v_{n-1} 
\end{bmatrix}\in \mathbb{R}^{nm}
\end{equation}
with arbitrary $v_1,\ldots,v_{n-1}\in\mathbb{R}^m$ is the kernel of $M$. At the same 
time only
\begin{equation}\begin{bmatrix}
\Phi_0[w](t) = \int\limits_0^t w(\tau) \,  \td \tau \\
\Phi_1[w](t) = 0 \\ \vdots \\ \Phi_{n-1}[w](t) = 0
\end{bmatrix} =
\begin{bmatrix}
v_0 \\ 0 \\ \vdots \\0
\end{bmatrix} 
\end{equation}
with an arbitrary $v_0\in\mathbb{R}^m$ can be reached. Thus $w$ is never mapped onto 
$\kernel{M}$ and therefore \eqref{eq:introduction_alt:operator} is injective though 
$M$ is not.