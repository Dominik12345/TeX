\subsection{Properties of the integral operator}
By setting $\hat{\tau} = (t-\tau)
$ and $f(\hat{\tau}) = w(t-\hat{\tau})$ the integrals in $I^t$ are simplified to
\begin{equation}
I^t(f) = \left( \frac{1}{k!} \int\limits_0^t \tau^k f(\tau) \, \td\tau 
\right)_{k\in\mathbb{N}_0} \quad .
\end{equation}

\begin{proposition}
The operator $I^t$ is bounded and continuous.
\end{proposition}
\begin{proof}
Estimation of the integrals yield
\begin{equation}
\begin{aligned}
|| I^t(f) ||_{l^2}^2 &= \sum\limits_{k=0}^\infty \left| \frac{1}{k!}
\int\limits_0^t  \tau^k f(\tau) \, \td \tau \right|^2 
\leq \sum\limits_0^\infty \left( \frac{1}{k!}\int\limits_0^t 
\left| \tau^k f(\tau) \right| \, \td\tau \right)^2 \\
&\leq \sum\limits_0^\infty \left( \frac{t^k}{k!}\int\limits_0^t 
\left|f(\tau) \right| \, \td\tau \right)^2 
\leq \left( \sum\limits_0^\infty  \frac{t^k}{k!}\int\limits_0^t 
\left|f(\tau) \right| \, \td\tau \right)^2 
= \left(\e^t ||f||_{L^1} \right)^2 \quad .
\end{aligned}
\end{equation}
Thus $||I^t(f)||_{l^2} \leq \e^t ||f||_{L^1}$ which means $I^t$ is 
bounded. Each bounded linear operator is 
continuous.
\end{proof}

\subsubsection{Fixed point of time}\label{integral:fixed}
In the following, $t>0$ is a fixed point of time. Considering $t$ as a variable will 
lead to different conditions, as shown in \ref{integral:variable}.
To prove that $I^t$ is injective we  
define the $n$-th integral function $F_n$ by
\begin{align}
F_{n+1}(\tau) &= \int\limits_0^\tau F_n(\tau') \,\td\tau' \tab{,}
F_0(\tau) = f(\tau) \\
F_n(0) &= 0 \tab{,} n=1,2,3,\ldots	
\end{align}
and use the following lemma.

\begin{lemma}\label{lemma:integral:roots}[Extreme Value Theorem with 
marginal conditions]\\
Let $g:[a,b]\to \mathbb{R}$ be a continuous function, $G$ such that 
$G'(\tau)=g(\tau)$ and $g(a)=g(b)=G(a)=G(b)=0$. Let $\{\tau_1,\tau_2,
\ldots, \tau_N\}$ be the set of roots of $g$ in $(a,b)$. Then $G$ has 
at most $N-1$ roots in $(a,b)$.
\end{lemma}
\begin{proof}
The roots of $g$ form a finite set hence there is no interval where $g$ 
is zero. Then $G$ has an local extremum or an saddle point at each $
\tau_i$, 
and thus at most one root in $(\tau_i,\tau_{i+1})$. Since $G$ cannot be 
zero in $(a,\tau_1)$ and $(\tau_N,b)$, $G$ has at most 
$N-1$ roots.
\end{proof}

In the proof of injectivity, $F_n$ will fit the assumptions of lemma 
\ref{lemma:integral:roots}. To see that, define the 
integral operators
\begin{equation}
I_{k,n}^t(f) = \frac{1}{(k-n)!}\int\limits_0^t \tau^{k-n} F_n(\tau) \, \td\tau \quad .
\end{equation}
We deduce a recursive formula 
\begin{equation}
I^t_{k,n}(f) = \frac{t^{k-n}}{(k-1)!}F_{n+1}(t) - I^t_{(k,n+1)}(f)  
\label{eq:integral:recursive}
\end{equation}
and identify 
\begin{equation}
\left(I^t_{k,0}(f)\right)_{k\in\mathbb{N}_0}=I^t(f) \quad . 
\label{eq:integral:identity}
\end{equation}
Combining \eqref{eq:integral:recursive} and 
\eqref{eq:integral:identity} leads to the closed form expression
\begin{equation}
I^t(f) = \left( \sum\limits_{n=0}^k (-1)^n\frac{t^{k-n}}{(k-n)!} 
F_{n+1}(t)
\right)_{k\in\mathbb{N}_0} \quad . \label{eq:integral:closed}
\end{equation}


\begin{proposition} \label{proposition:integral:injective_fixed}
The operator $I^t$ is injective.
\end{proposition}
\begin{proof}
Let $I^t(f)=0 \in l^2$. With \eqref{eq:integral:closed} it follows 
that
\begin{equation}
F_n(t) = 0 \tab{,} n=1,2,3,\ldots \quad . \label{eq:integral:zero}
\end{equation}
Let $\mathcal{R}$ be the countable set where $f(\tau) = 0$ for all $
\tau\in\mathcal{R}$, let $\mathcal{S}$ be the set where $f$ is not 
continuous an changes sign and let $\mathcal{I}$ be the set of 
intervals where $f$ is zero.
$F_1$ cannot have a local extremum on any interval of 
$\mathcal{I}$. Thus
\begin{equation}
\tau^*\quad \text{is an local extremum} \quad \Rightarrow \quad 
\tau^* \in \mathcal{R}\cup\mathcal{S}
\end{equation} 

	\begin{enumerate}
	\item If $\mathcal{R}\cup \mathcal{S}$ is finite, then 
	$F_{|\mathcal{R}|+|\mathcal{S}|+1}$ surely has no local extremum 
	and 
	due to \eqref{eq:integral:zero}, $F_{|\mathcal{R}|+|
	\mathcal{S}|+1}$ has to be the null function. Because of 
	$F_n'(\tau) = 
	F_{n-1}(\tau)$, each $F_n$ is the null function and $f=0$ a.e.
	\item If $\mathcal{R}\cup \mathcal{S}$ is infinite then either $f$ 
	is not in $L^1([0,t])$ or $\mathcal{R}\cup \mathcal{S}$ can be 
	split into a finite part and a null set.
	\end{enumerate}
\end{proof}


\subsubsection{Variable time}\label{integral:variable}
\label{proposition:integral:injective}
Now we consider the full information of \eqref{eq:introduction:observable}, i.e. 
$I^t$ maps a function onto a trajectory in $l^2$. 
\begin{proposition}
	The operator $I^t$ is injective.
\end{proposition}
\begin{proof}
	Let $f$ such that $I^t=0\in l^2$ for each $t\in[0,T]$.
	\begin{itemize}
	\item Setting $t=T$ and using \ref{proposition:integral:injective_fixed} proves that 
	$f=0$ a.e.
	\item In general consider a function $f$ that is not zero a.e. but possibly $f(t)=0$ 
	on $[0,t_1]$. Chose $t_2>t_1$ to be the first 
	point where $f$ changes sign. Then each component of $I^t_2(f)$ is non zero. Thus $f$ 
	has to be zero 
	on $[t_1,t_2]$ a.e. Repeat this argument until $f=0$ on $[0,T]$ a.e.
	\end{itemize}
\end{proof}
The two ways to prove the preceding proposition show, that a variable time puts more 
much stronger constrains to the hidden input $f$.