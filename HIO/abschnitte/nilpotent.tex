\subsubsection{Nilpotent Dynamics}
Let $A$ be a nilpotent matrix, i.e. there is a regular $n\times n$ matrix $P$ such that
\begin{equation}
 A = P^{-1} A_\triangle P
\end{equation}
with $A_{\triangle \omega \mu} = 0$ when $\omega \leq \mu$. As a graphical condition this 
means, that $A$ can be represented by a directed acyclic graph. This yields
\begin{equation}
y(t) = \sum\limits_{l=0}^{n-1} \underbrace{CP^{-1}}_{\rank{CP^{-1} 
=\rank{C}} } A^l_\triangle \Phi_l[\underbrace{Pw}_\text{bijection} ](t) \quad .
\end{equation}
Thus without loss of generality we can assume that $A$ is strictly lower 
triangular. Furthermore we see that \eqref{eq:HIO:Phi} reduces to
\begin{equation}
\Phi_l[w_\mu](t) = \int\limits_0^t \frac{(t-\tau)^l}{l!} w_\mu(\tau) \, \td\tau \quad .
\label{eq:HIO:Phi_nil}
\end{equation}

\begin{lemma}[Without proof]
The operators defined by \eqref{eq:HIO:Phi_nil} have the properties
\begin{equation}
\frac{\td}{\td t} \Phi_l[w_\mu](t) = \Phi_{l-1}[w_\mu](t)  
\tab{and} \frac{\td}{\td t} \Phi_0[w_\mu](t) = w_\mu(t)
\quad .
\end{equation}
\end{lemma}

\begin{proposition}
The operators $\{\Phi_l|l\in\{0,1,\ldots,n-1\}\}$ from a nilpotent matrix form an injective 
set.
\end{proposition}
\begin{proof}
Let $\{v_{l}\}$ be a set of functions with $l\in\mathcal{L}=\{0,1,\ldots,n-1\}$. Set
\begin{equation}
\sum\limits_{l\in\mathcal{L}} \Phi_l[v_l] 
\equiv 0 \quad . 
\end{equation}
Writing this as integral equation
\begin{equation}
\int\limits_0^t \sum\limits_{l\in\mathcal{L}}  
 \frac{(t-\tau)^l}{l!} v_l(\tau) \, \td \tau = 0 \quad \forall t\in[0,T]
\end{equation}
which means
\begin{equation}
 \sum\limits_{l\in\mathcal{L}}  
\frac{(t-\tau)^l}{l!} v_l(\tau) = 0 \quad \forall (t,\tau)
\in [0,T]\times [0,t] \quad .
\end{equation}
Now let $l_\text{min}$ be the smallest $l$ in $\mathcal{L}$. This leads to 
\begin{equation} 
 \frac{1}{l_\text{min}!} v_l(\tau) = - 
\sum\limits_{l_\text{min} <l\in\mathcal{L}}  
\frac{(t-\tau)^{l-l_\text{min}}}{l!}  v_l(\tau) \quad .
\end{equation}
Since the left hand side of this equation is independent from $t$, so must the 
right hand side. Evaluating the derivatives with respect to $t$ leads to 
\begin{equation}
v_l(\tau) = 0 \quad \forall \tau \in [0,T]
\end{equation}
for all $l\in \mathcal{L}$ separately. This means $\{I_l\}$ is an injective 
set.
\end{proof}