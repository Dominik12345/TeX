\section{Properties of the matrix series}
A key statement is the Cayley-Hamilton theorem. A proof can be found in 
\cite{Greub}.
\begin{theorem}[Cayley-Hamilton]
	Let $A$ be an $n\times n$ matrix and $\sum_{k=0}^n \tilde{c}_k 
	\lambda^k $ 
	the characteristic polynomial. Then
	\begin{equation}
	\sum\limits_{k=0}^n \tilde{c}_k A^k = 0 \quad .
	\end{equation}
\end{theorem}
\begin{corollary}
	\begin{enumerate}
	\item Each $A^N$, $N\geq n$, can be written as $\sum_{k=0}^{n-1} 
	c_k A^k$ with some coefficients $\{c_k\}$ that can be 
	calculated from $\{\tilde{c}_k\}$.
	\item Multiplying with $C$ and $D$ also shows 
	\begin{equation}
	CA^ND = \sum\limits_{k=0}^{n-1} c_k C A^k D
	\end{equation}
	for any $N\geq n$.
	\item For each operator $\Sigma_N (s_k)_{k\in\mathbb{N}_0}= \sum_{k=0}^{N-1} CA^kD 
	(s_k)_{k\in\mathbb{N}_0} $ with $N
	\geq 0$ 
	\begin{equation}
	\rank{\Sigma_N} = \rank{\Sigma_n} \quad .
	\end{equation}	 
	\end{enumerate}
\end{corollary}
\begin{lemma}
	The operator series $\Sigma_N$ converges in norm and $\Sigma_N \longrightarrow \Sigma
	$.
\end{lemma}
\begin{proof}
	The operator norm is defined by
	\begin{equation}
	||\Sigma_N|| := \supremum{|| \Sigma_N (s_k) ||}{||(s_k)||_{l^2} = 1} \quad .
	\end{equation}
\end{proof}


Again, considering $t>0$ as a fixed point in time and assuming $I^t$ is surjective. We 
use the shorter notation $(s_k)$ for a series $(s_k)_{k\in\mathbb{N}_0}\in l^2$.
\begin{theorem}
	The kernel of $\Sigma$ is infinite dimensional.
\end{theorem}
\begin{proof}
	Define a $l^2$ series by $s_k=\delta_{k,N}$ for a $N\geq n$. Then
	\begin{equation}
	\Sigma (s_k) = CA^nD = \sum\limits_{k=0}^{n-1} c_kCA^kD \quad .
	\end{equation}
	The $l^2$ series $\hat{s}_k = c_k$ for $k < n$ and $\hat{s}_k = 0$ for $k\geq n$ 
	leads to the same result, thus 
	\begin{equation}
	\linspan{(\hat{s}_k-s_k)} \subset \kernel{\Sigma} \quad .
	\end{equation}
	Repeating this argument for each $N\geq n$ it is possible to get arbitrarily many 
	linearly independent vectors $(\hat{s}_k-s_k)$.
\end{proof}

Now let $(s_k)(t)$ be a trajectory in $l^2$. 
