\section{HIO using Volterra-operators}
Let $w:[0,T]\rightarrow \mathbb{R}^m$, $y:[0,T]\rightarrow \mathbb{R}^p$, 
$A\in \mathbb{R}^{n\times n}$, $D\in\mathbb{R}^{n\times m}$ and $C\in\mathbb{R}^{p\times n}
$ such that 
\begin{equation}
y(t) = \int\limits_0^t C \exp{\left\{A(t-\tau)\right\}} D w(\tau) \, \td \tau \quad .
\label{eq:volterra:y}
\end{equation}
Due to Cayley-Hamilton
\begin{equation}
A^k = \sum\limits_{l=0}^{n-1} c_{k,l} A^l \label{eq:volterra:cayley-hamilton}
\end{equation}
with coefficients $c_{k,l}$ that in general are not unique. By choosing $N$ the 
smallest number such that
\begin{equation}
A^N \in \linspan{\left(A^0,A^1,\ldots,A^{N-1} \right)} \quad ,
\end{equation}
the coefficients $c_{k,l}$ count $l=0,1,\ldots,N-1$ and are unique. For each $k\leq 
N-1$ we find the Kronecker-delta
\begin{equation}
c_{k,l} = \delta_{k,l} \quad . \label{eq:volterra:cayley-hamilton_unique}
\end{equation}

Expanding the exponential function to its power series we get
\begin{equation}
\begin{aligned}
\exp\{ A(t-\tau) \} &= \sum\limits_{k=0}^\infty \frac{(t-\tau)^k}{k!} A^k \\
%&= \sum\limits_{k=0}^\infty \frac{(t-\tau)^k}{k!} \sum\limits_{l=0}^{n-1} c_{k,l} A^l \\
&= \sum\limits_{l=0}^{N-1}  A^l \underbrace{\sum\limits_{k=0}^\infty \frac{(t-\tau)^k}
{k!}  
c_{k,l}}_{\phi_l(t,\tau)} \quad .
\end{aligned} \label{eq:volterra:sums}
\end{equation}
Both sums should converge absolutely since they are either finite or suppressed 
exponentially. Using \eqref{eq:volterra:sums} we can write \eqref{eq:volterra:y} as
\begin{equation}
y(t) = \sum\limits_{l=0}^{n-1} CA^lD \int\limits_0^t \phi_l(t,\tau) w(\tau) \, \td\tau
\quad . \label{eq:volterra:y_volterra}
\end{equation}

\subsection{Scalar Volterra-operator}
\begin{definition}
According to \cite{Heuser} or \cite{Kirsch} the equation
\begin{equation}
\int\limits_0^t K(t,s) f(s) \, \td s =  g(t) \label{eq:volterra:definition}
\end{equation}
where $f,g:[0,T] \rightarrow \mathbb{R}$ and $K:\{ (t,s)\in [0,T]\times [0,T] | 
s<t\}\rightarrow \mathbb{R}$ is called \textit{Volterra integral equation of the 
first kind (V1)} and $K$ is called the \textit{kernel}.
\end{definition}

Comparing \eqref{eq:volterra:y_volterra} and \eqref{eq:volterra:definition}
we see that, if $w_i$ denotes the $i$-th component of $w$, we get $m$ V1 equations
\begin{equation}
\int\limits_0^t \phi_l(t,\tau) w_i(\tau) \, \td\tau = z_i(t) \quad .
\label{eq:volterra:solve_z}
\end{equation}
Let $z(t)$ be any desired trajectory in $\mathbb{R}^m$ and we want to solve 
\eqref{eq:volterra:solve_z}. 

\begin{definition}
Again according to \cite{Heuser} and \cite{Kirsch}
\begin{equation}
f(s)-\int\limits_0^t K(t,s) f(s) \, \td s = g(t)
\end{equation}
is called \textit{Volterra integral equation of the second kind (V2)}.
\end{definition}
\begin{theorem}[Proof in \cite{Heuser} and \cite{Kirsch}]
For any admissable function $g$, V2 has a uni\-que solution $f$ that is given by the 
Neumann series.
\end{theorem}
\begin{theorem}\label{theorem:volterra:V1}
Let $K$ be a continuous kernel, $K(t,t)\neq 0$ and 
$\frac{\partial K}{\partial t}$ continuous. Then, given an differentiable $g$, V1 has a 
unique solution $f$.
\end{theorem}
\begin{proof}
We follow the proof in \cite{Kirsch}.\\
Differentiation of V1 with respect to $t$ yields
\begin{equation}
K(t,t)f(t) + \int\limits_0^t \frac{\partial K}{\partial t}(t,s) f(s) \, \td s = \frac{\td 
g}{\td t}(t) \quad
\end{equation}
Only if $K(t,t)\neq 0$ we can divide the whole equation by $K(t,t)$ and redefine the 
Kernel and right hand side to end up with a V2, which has a unique solution.
\end{proof}

Now turning back to \eqref{eq:volterra:solve_z}, we want to apply theorem 
\ref{theorem:volterra:V1} to get information about the solution. But we find that 
the kernels $\phi_l$ do not fit to the assumption since
\begin{equation}
\phi_l(t,t) = \sum\limits_{k=0}^\infty \frac{(t-t)^k}{k!} c_{k,l} = c_{0,l}
\end{equation}
and comparing to \eqref{eq:volterra:cayley-hamilton_unique} 
\begin{equation}
\phi_0(t,t) = 1 \tab{,} \phi_l (t,t) = 0 \tab{for}l=1,2,\ldots,N-1 \quad .
\end{equation}