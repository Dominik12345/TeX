\section{Hidden Input Observability}
	Consider a dynamic system $\mathcal{S}$
	\begin{align}
	\frac{\td x_w}{\td t} &= Ax_w(t) + Bu(t) + Dw(t) \\
	y_w(t) &= Cx_w(t) \\
	x_w(0) &= x_0
	\end{align}
	where $x_w$, $u$, $w$ and $y_w$ map $[0,T]$ onto $\mathbb{R}^n$, 
	$\mathbb{R}^{\hat{m}}$, $\mathbb{R}^m$ and $\mathbb{R}^p$, respectively, and 
	$A$, $B$, $C$, $D$ are matrices of suitable dimensions. We assume the function $u$ is
	known and called the \textit{known input}, the function $w$ is unknown called 
	 \textit{hidden input}. The closed form solution for $y_w$ is
	\begin{equation}
	y_w(t) = C \int\limits_0^t \exp(A(t-\tau)) (Bu(\tau) + Dw(\tau) ) \, \td 
	\tau \quad .
	\end{equation}

\begin{definition}{Hidden Input Observability}{}
If for $\mathcal{S}$ the implication
\begin{equation}
y_{w}(t) = y_{\hat{w}}(t) \quad \forall t\in [0,T]
\quad \Rightarrow \quad 
 w = \hat{w} \quad \text{a.e.}
\end{equation}
holds, $\mathcal{S}$ is called \textit{hidden input observable (HIO)}. If this 
implication holds only for 
$m-m'$ components of $w$, $\mathcal{S}$ is 
called \textit{HIO with $m'$ limitations}.
\end{definition}

Our aim is to find necessary or sufficient conditions for the hidden input observability 
of linear systems.\\

Due to linearity, $\mathcal{S}$ is HIO if and only if
\begin{equation}
y(t):=C\int\limits_0^t \exp(A(t-\tau))Dw(\tau)\,\td\tau = 0 \quad 
\forall t\in[0,T] \quad \Rightarrow \quad 
w = 0 \quad \text{a.e.} \label{eq:HIO:y}
\end{equation}

We firstly rearrange this equation using the operator formalism and then a\-na\-lyse its 
properties.

\clearpage
\subsubsection*{Rearranging the equation}
	By the Cayley-Hamilton theorem, for any $k\in\mathbb{N}_0$ there are coefficients 	
	$c_{k,l}$ 
	such that 
	\begin{equation}
	A^k = \sum\limits_{l=0}^{n-1} c_{k,l} A^l \quad .\label{eq:HIO:cayley-hamilton}
	\end{equation}
	Defying
	\begin{equation}
	\phi_k[w](t) := \int\limits_0^t \frac{(t-\tau)^k}{k!} w(\tau) \, \td \tau 
	\tab{and}
	\Phi_l[w](t) :=  \sum\limits_{k=0}^\infty c_{k,l} \phi_k[w](t)  \label{eq:HIO:Phi}
	\end{equation}
	equation \eqref{eq:HIO:y} can be written as
	\begin{equation}
	y(t) = \sum\limits_{l=0}^{n-1} CA^lD \, \Phi_l[w](t)
	\quad . \label{eq:HIO:y_operator}
	\end{equation}

\subsubsection*{Basic properties}
%\begin{lemma}[Without proof] \label{prop:HIO:Phi_injective}
%	Each operator $\Phi_l$ defined by \eqref{eq:HIO:Phi_nil} is injective, i.e. 
%	\begin{equation}
%	\Phi_l[w] \equiv 0\quad \quad \Rightarrow  \quad w \equiv 0  
%	\end{equation}
%	and $\Phi_0$ is surjective.
%	Here "$\equiv$" denotes equality to the zero function and $\Phi_l$ operates 
%	component-wise on $(w_1,w_2,\ldots,w_m)^\text{T}:[0,T]\rightarrow \mathbb{R}^m$.
%\end{lemma}

%\begin{definition}{Injective Set}{}
%	Let $\mathcal{L}$ be an index set. A set 
%	\begin{equation}
%	\left\{\quad\Phi_l:L^2([0,T])\to L^2([0,T])\quad|\quad l\in\mathcal{L}\quad \right\}
%	\end{equation}
%	of operators is called 
%	\textit{injective set}, if for a set of functions
%	$\{v_l \in L^2([0,T])|l\in\mathcal{L}\}$
%	the implication 
%	\begin{equation}
%	\sum\limits_{l\in\mathcal{L}} 
%	\Phi_l\left[v_l \right] \equiv 0 \quad \Rightarrow \quad  
%	v_l \equiv 0 \quad \forall l\in \mathcal{L}
%	\end{equation}
%	holds.
%\end{definition}

%\begin{lemma}{(Without proof)}{} \label{lemma:Volterra1}
%	As a Volterra integral equation of the first kind the implication
%	\begin{equation}
%	\int\limits_0^t f(\tau) \,\td\tau = 0 \quad \forall t\in[0,T] \quad \Rightarrow \quad
%	f(t) = 0\quad \forall t\in[0,T]
%	\end{equation}
%	holds.
%\end{lemma}

%\begin{corollary}[Without proof]
%	If $\{\Phi_l|l\in\mathcal{L}\}$ is an injective set, then each $\Phi_l$ is injective.
%\end{corollary}

%\begin{proposition}{}{} \label{prop:phi_injectiveset}
%	The operators $\{\phi_k|k\in\mathbb{N}_0\}$ defined by \eqref{eq:HIO:Phi} form an 
%	injective set.
%\end{proposition}

%\begin{proof}
%Let $\{v_{l}\}$ be a set of functions with $l\in\mathcal{L}=\{0,1,\ldots,n-1\}$. Set
%\begin{equation}
%\sum\limits_{l\in\mathcal{L}} \Phi_l[v_l] 
%\equiv 0 \quad . 
%\end{equation}
%Writing this as integral equation
%\begin{equation}
%\int\limits_0^t \sum\limits_{l\in\mathcal{L}}  
% \frac{(t-\tau)^l}{l!} v_l(\tau) \, \td \tau = 0 \quad \forall t\in[0,T]
%\end{equation}
%which means
%\begin{equation}
% \sum\limits_{l\in\mathcal{L}}  
%\frac{(t-\tau)^l}{l!} v_l(\tau) = 0 \quad \forall (t,\tau)
%\in [0,T]\times [0,t] \quad .
%\end{equation}
%Now let $l_\text{min}$ be the smallest $l$ in $\mathcal{L}$. This leads to 
%\begin{equation} 
% \frac{1}{l_\text{min}!} v_l(\tau) = - 
%\sum\limits_{l_\text{min} <l\in\mathcal{L}}  
%\frac{(t-\tau)^{l-l_\text{min}}}{l!}  v_l(\tau) \quad .
%\end{equation}
%Since the left hand side of this equation is independent from $t$, so must the 
%right hand side. Evaluating the derivatives with respect to $t$ leads to 
%\begin{equation}
%v_l\equiv 0
%\end{equation}
%for all $l\in \mathcal{L}$ separately. This means $\{I_l\}$ is an injective 
%set.
%\end{proof}

%\begin{proof}	
%	Let $\{v_k|k\in\mathbb{N}_0\}$ be a set of functions and set
%	\begin{equation}
%	\sum\limits_{k=0}^\infty \phi_k[v_k] \equiv 0 \quad .
%	\end{equation}
%	Inserting the definition of $\phi_k$ and using the lemma \ref{lemma:Volterra1} we 
%	can deduce
%	\begin{equation}
%	\sum\limits_{k=0}^\infty \frac{(t-\tau)^k}{k!} v_k(\tau) = 0 \quad \forall (t,\tau)
%	\in [0,T]\times (0,t)  \quad .
%	\end{equation}
%	Choose $\tau=T-\epsilon$ with a small $\epsilon>0$ to get
%	\begin{equation}
%	v_0(t) + \mathcal{O}(\epsilon) = 0 \quad \forall t\in [0,T]
%	\end{equation}
%	and when $\epsilon\to 0$ we get $v_0\equiv 0$. Dividing the preceding equation by 
%	$\epsilon$ we get
%	\begin{equation}
%	v_1(t) + \mathcal{O}(\epsilon) = 0 \quad \forall t\in [0,T]
%	\end{equation}
%	which leads to $v_1\equiv 0$. Proceeding in a similar manner proves the lemma.
%\end{proof}
	
%\begin{lemma}{}{}
%	The series $(c_{k,l})_{k\in\mathbb{N}_0}$ from \eqref{eq:HIO:cayley-hamilton} can 
%	be chosen in such a way that they are linearly independent, i.e. for coefficients  
%	$d_l$ and for all $k\in\mathbb{N}_0$
%	\begin{equation}
%	\sum\limits_{l\in\mathcal{L}} d_l c_{k,l} = 0 \quad \Rightarrow \quad
%	d_l=0\quad \forall l \quad .
%	\end{equation}
%\end{lemma}
%\begin{proof}
%	In the Cayley-Hamilton theorem \eqref{eq:HIO:cayley-hamilton} we can choose $c_{k,l}=
%	\delta_{kl}$ for $k\in\{0,1,\ldots,n-1\}$ without issues. Then, by comparing the 
%	first components of $c_{.,l}$ we see that the series are linearly independent.   
%\end{proof}

%\begin{proposition}{}{} \label{prop:HIO:always_injective}
%	The operators $\{\Phi_l|l\in\{0,1,\ldots,n-1\}\}$ defined by \eqref{eq:HIO:Phi} and 
%	with coefficients from the Cayley-Hamilton theorem \eqref{eq:HIO:cayley-hamilton} 
%	always form an injective set.
%\end{proposition}
%\begin{proof}
%	Using the preceding lemma, without loss of generality
%	\begin{equation}
%	\Phi_l = \phi_l + \sum\limits_{k=n}^\infty c_{k,l} \phi_k \quad .
%	\end{equation}
%	Now using that $\phi_l$ are linear operators
%	\begin{equation}
%	\sum\limits_{l=0}^{n-1} \Phi_l[v_l] = \sum\limits_{k=0}^{n-1} \phi_k[v_k] + 
%	\sum\limits_{k=n}^\infty \phi_k\left[\sum\limits_{l=0}^{n-1} c_{k,l}v_l \right] 
%	\equiv 0
%	\end{equation}
%	where, using proposition \ref{prop:phi_injectiveset}, we conclude $v_k\equiv 0$ for 
%	$k\in\{0,1,\ldots,n-1\}$. This is sufficient to complete the proof.
%\end{proof}
\begin{lemma}{}{}
	For the operators $\phi_k$ defined by \eqref{eq:HIO:Phi} we find
	\begin{equation}
	\frac{\td}{\td t} \phi_k[w] = \phi_k[w] \tab{and} \frac{\td}{\td t} \phi_0[w]=w
	 \quad .
	\end{equation}
\end{lemma}
\begin{proof}
	Consider
	\begin{equation}
	\frac{\td}{\td t} \phi_k[w](t) := \lim_{\Delta t \to 0} \frac{\phi_k[w](t+\Delta t)
	- \phi_k[w](t)}{\Delta t} \quad .
	\end{equation}
	When $k > 0$
	\begin{align}
	\phi_k[w](t+\Delta t) &= \int\limits_0^{t} \frac{(t+\Delta t-\tau)^k}{k!} w(\tau) \, 
	\td \tau 
	+ \int\limits_t^{t+\Delta t} \frac{(t+\Delta t-\tau)^k}{k!} w(\tau) \, \td \tau \\
	&= \phi_k[w](t) + \Delta t \, \phi_{k-1}[w](t) 
	+ \mathcal{O}(\Delta t^2)
	\end{align}
	and when $k=0$
	\begin{equation}
	\phi_k[w](t+\Delta t)  = \phi_k[w](t) + \Delta t \, w(t) + \mathcal{O}(\Delta t^2)
	\end{equation}
	and inserting into the definition of $\nicefrac{\td}{\td t}\phi_k[w]$ completes the 
	proof.
\end{proof}

\begin{lemma}{}{} \label{prop:HIO:phi_injective}
	Each operator $\phi_k$ defined by \eqref{eq:HIO:Phi} is injective.
\end{lemma}
	\begin{proof}
	Inductive proof:
	\begin{enumerate}
	\item As a Volterra operator of the first kind with kernel $1$, $\phi_0$ is 
	injective.
	\item Assume $\phi_k$ is injective. Set $\phi_{k+1}[w](t)\equiv 0$. Then
	\begin{equation}	 
	\frac{\td}{\td t}\phi_{k+1}[w](t)=\phi_k[w](t)\equiv 0
	\end{equation}	
	 which implies
	 $w\equiv 0$ since $\phi_k$ is injective.
	\end{enumerate}
	\end{proof}