\section{HIO using Volterra-operators and linearly independent hidden inputs }
Let $w:[0,T]\rightarrow \mathbb{R}^m$, $y:[0,T]\rightarrow \mathbb{R}^p$, 
$A\in \mathbb{R}^{n\times n}$, $D\in\mathbb{R}^{n\times m}$ and $C\in\mathbb{R}^{p\times 
n}
$ such that 
\begin{equation}
y(t) = \int\limits_0^t C \exp{\left\{A(t-\tau)\right\}} D w(\tau) \, \td \tau \quad .
\label{eq:volterra:y}
\end{equation}
Due to Cayley-Hamilton
\begin{equation}
A^k = \sum\limits_{l=0}^{n-1} c_{k,l} A^l 
\end{equation}
with coefficients $c_{k,l}$ that in general are not unique. By choosing $N\leq n$ the 
smallest number such that
\begin{equation}
A^N \in \linspan{\left(A^0,A^1,\ldots,A^{N-1} \right)} \quad ,
\end{equation}
the coefficients $c_{k,l}$ count $l=0,1,\ldots,N-1$ and are unique. 
Expanding the exponential function to its power series we get
\begin{equation}
y(t) = \sum\limits_{l=0}^{N-1} CA^lD \Phi_l[w](t)
\quad . \label{eq:volterra_alt1:y_volterra}
\end{equation}
where
\begin{equation}
\Phi_l[w](t) :=  \int\limits_0^t \sum\limits_{k=0}^\infty \frac{(t-\tau)^k}{k!} c_{k,l} 
w(\tau) \, \td\tau \quad .
\label{eq:volterra_alt1:Phi_l}
\end{equation}
\begin{proposition}[Without proof] \label{prop:volterra_alt1:1}
Each operator $\Phi_l$ is injective, i.e. 
\begin{equation}
\Phi_l[w] \equiv 0\quad \quad \Rightarrow  \quad w \equiv 0  
\end{equation}
and $\Phi_0$ is surjective.
Here "$\equiv$" denotes equality to the zero function and $\Phi_l$ operates component-
wise on $(w_1,w_2,\ldots,w_m)^\text{T}:[0,T]\rightarrow \mathbb{R}^m$.
\end{proposition}
%\clearpage

Consider the simple case $D=\mathbb{1}$ ($m=n$). Then for the $\mu$-th column of $CA^k$ 
we find
\begin{equation}
\left(CA^k \right)_\mu = \sum\limits_{\omega = 1}^n A^k_{\omega\mu} C_\omega
\end{equation}
where $A^k_{\omega\mu}$ is the $(\omega\mu)$ component of $A^k$ and $C_\omega$ is the 
$\omega$-th column of $C$.
Therefore each column of any $CA^k$ is a linear combination of column vectors 
$C_\omega$ and thus 
\begin{equation}
\rank{\left[C,CA,CA^2,\ldots,CA^{N-1}\right]} = \rank{C} \quad .
\end{equation}

Furthermore we assume the hidden inputs to be  
linearly independent, i.e. for any coefficients $d_\mu$
\begin{equation}
\sum\limits_{\mu=1}^n d_\mu w_\mu \equiv 0 \quad \Longleftrightarrow \quad 
d_\mu w_\mu \equiv 0 \quad \forall \mu \quad . 
\end{equation}

\subsection{Nilpotent dynamics}
Let $A$ be a nilpotent matrix, i.e. there is a regular $n\times n$ matrix $P$ such that
\begin{equation}
 A = P^{-1} A_\triangle P
\end{equation}
with $A_{\triangle \omega \mu} = 0$ when $\omega \leq \mu$. As a graphical condition this 
means, that $A$ can be represented by a directed acyclic graph. Inserting in  
\eqref{eq:volterra_alt1:y_volterra} yields
\begin{equation}
y(t) = \sum\limits_{l=0}^{N-1} \underbrace{CP^{-1}}_{\rank{CP^{-1} 
=\rank{C}} } A^l_\triangle \Phi_l[\underbrace{Pw}_{ \text{lin.indep.}}](t) \quad .
\end{equation}
Thus without loss of generality we can assume that $A$ is strictly lower 
triangular and $N=n$. Furthermore we see that \eqref{eq:volterra_alt1:Phi_l} reduces to
\begin{equation}
\Phi_l[w_\mu](t) = \int\limits_0^t \frac{(t-\tau)^l}{l!} w_\mu(\tau) \, \td\tau
\end{equation}
with the properties
\begin{equation}
\frac{\td}{\td t} \Phi_l[w_\mu](t) = \Phi_{l-1}[w_\mu](t)  
\tab{and} \frac{\td}{\td t} \Phi_0[w_\mu](t) = w_\mu(t)
\quad , \label{eq:volterra_alt1:Phi_rules}
\end{equation}
and equation \eqref{eq:volterra_alt1:y_volterra} becomes
\begin{equation}
y(t) = \sum\limits_{\omega=1}^{n} \sum\limits_{l=0}^{\omega-1}  
\sum\limits_{\mu=1}^{\omega-l} 
 \Phi_l\left[A^l_{\omega\mu} w_\mu \right](t)  C_\omega 
=: \sum\limits_{\omega=1}^n \varphi_\omega (t) C_\omega 
 \quad . \label{eq:volterra_alt1_varphi}
\end{equation}
From \eqref{eq:volterra_alt1:Phi_rules} one can also deduce
\begin{equation}
\left(\frac{\td}{\td t}\right)^q \varphi_\omega = \sum\limits_{l=q}^{\omega-1} 
\sum\limits_{\mu=1}^{\omega-l} \Phi_{l-q}\left[A^l_{\omega\mu} w_\mu \right]
+\sum\limits_{l=0}^{q-1} \sum\limits_{\mu=1}^{\omega-l} A^l_{\omega\mu} 
w_\mu^{(q-l-1)} \label{eq:volterra_alt1:first_derivative}
\end{equation}
where $w_\mu^{(q)}$ denotes the $q$-th derivative of $w_\mu$. Useful derivatives are
\begin{equation}
\frac{\td}{\td t} \varphi_\omega = \sum\limits_{l=1}^{\omega-1} 
\sum\limits_{\mu=1}^{\omega-l} \Phi_{l-1}\left[A^l_{\omega\mu} w_\mu \right]
+ w_\omega
\tab{,}
\left( \frac{\td}{\td t}\right)^\omega \varphi_\omega= 
\sum\limits_{l=0}^{\omega-1} \sum\limits_{\mu=1}^{\omega-l} A^l_{\omega\mu} 
w_\mu^{(\omega-l-1)} \quad .
\end{equation}
\begin{corollary}[Without proof]\label{corollary:volterra_alt1:M}
From \eqref{eq:volterra_alt1:first_derivative} with $q=1$ we also conclude that if there 
is a $M$ 
such that $w_\mu \equiv 0$ for all $\mu=1,2,\ldots,M-1$, then $\nicefrac{\td}{\td t}
\, \varphi_M = w_M$.
\end{corollary}

%\subsubsection{$C$ has full column rank $n$}\label{subsub1}
%	Assume the columns of $C$, $\{C_1,C_2,\ldots,C_n\}$ form a linearly independent set 
%	of $\mathbb{R}^p$ vectors. Setting $y\equiv 0$ and equating coefficients we 
%	get $\varphi_\omega\equiv 0\forall \omega$ and by the preceding corollary
%	\begin{equation}
%	 \frac{\td}{\td t} \varphi_\omega =  w_\omega  \equiv 0
%	\end{equation}
%	showing, that such a system is HIO.
%%	\begin{example}
%%	Consider the system
%%	\begin{equation}
%%	A =\begin{pmatrix}
%%	 2 & -1 & -1 & 0 \\ 2 & -1 & -1 & \nicefrac{1}{2} \\
%%	1 & 0 & -\nicefrac12 & -\nicefrac34\\ -2 & 2 & 1 & -\nicefrac12
%%	\end{pmatrix} \quad 
%%	C = \begin{pmatrix}
%%	1 & 0 & 0 & 0 \\ 1 & 1 & 1 & 0  \\ 0& 1 & 1 & 1\\ 0 &1 &1 &0\\ 1 &0&1&1
%%	\end{pmatrix} \quad .
%%	\end{equation}
%%	With the matrix
%%	\begin{equation}
%%	P = \begin{pmatrix}
%%	1 & -\nicefrac12 & -\nicefrac12 &\nicefrac14\\ 0 & \nicefrac12 & 0 &0\\ 0&0&0&
%%	\nicefrac12\\ 0& 0&\nicefrac12&-\nicefrac14
%%	\end{pmatrix}
%%	\end{equation}		
%%	we transform the system to
%%	\begin{equation}
%%	A_\triangle = \begin{pmatrix}
%%	0 & 0 &0 &0\\1&0&0&0\\-1&1&0&0\\1&0&-1&0
%%	\end{pmatrix}		\quad
%%	\hat{C} = \begin{pmatrix}
%%	1 & 1 & 0 & 1\\1&3&1&3\\0&2&3&2\\0&2&1&2\\1&1&3&3
%%	\end{pmatrix} \quad ,
%%	\end{equation}
%%	to see that this system is hidden input observable.
%%	\end{example}
%	\begin{remark}
%	\begin{itemize}
%	\item To have a set $\{C_1,\ldots,C_n\}$ of linearly independent vectors, its crucial 
%	to have $n\leq p$.
%	\item A nilpotent matrix has only one Eigenvalue $0$. Thus also $\trace{A} = 0$ and 
%	$\det{A}=0$ are necessary conditions.
%	\end{itemize}
%	\end{remark}
%	
%\subsubsection{$C$ has column rank $n-1$}\label{subsub2}
%	Consider the case where $\{C_2,\ldots,C_n\}$ are linearly independent and 
%	$C_1=\sum_{\lambda=2}^n \Lambda_\lambda C_\lambda$ with coefficients 
%	$\Lambda_\lambda$. Then
%	\begin{equation}
%	y(t) = \sum\limits_{\omega=2}^n \left(\varphi_\omega (t) + \Lambda_\omega 
%	\varphi_1(t)\right)
%	C_\omega \quad .
%	\end{equation}
%	Again setting $y\equiv 0$ and equating coefficients yields
%	\begin{equation}
%	\frac{\td}{\td t} (\varphi_\omega+\Lambda_\omega \varphi_1) =
%	\sum\limits_{l=1}^{\omega-1}\sum\limits_{\mu=1}^{\omega-l} \Phi_{l-1}\left[
%	A^l_{\omega\mu} w_\mu \right] + w_\omega + \Lambda_\omega w_1 \equiv 0
%	\label{eq:volterra_alt1:C2}
%	\end{equation}
%	\begin{itemize}
%	\item If for any $\omega=2,3,\ldots,n$ we have $\Lambda_\omega\neq 0$ and 
%	$A^1_{\omega\mu}=0\forall \mu$ (and hence\\ $A^l_{\omega\mu}=0\forall\mu ,l\geq 1$), 
%	by linearly independence of $\{w_1,w_\omega\}$ it follows that $w_1\equiv 0$ and 
%	again by corollary \ref{corollary:volterra_alt1:M} the system is HIO. \\
%	For example, $\omega=2$ shows $A^1_{21}=0$ is sufficient.
%	\end{itemize}
%		
%	Now consider the case, that $C_1$ cannot be written as a linear combination of 
%	$\{C_2,\ldots,C_n\}$. Then for each $H=2,3,\ldots,n$ and $C_H=\sum_{\lambda=1,\lambda
%	\neq H}^n \Lambda_\lambda C_\lambda$ it follows $\Lambda_1 = 0$. We get
%	\begin{equation}
%	y(t) = \sum\limits_{\substack{\omega= 1 \\ \omega\neq H}}^n(\varphi_\omega+\Lambda_
%	\omega\varphi_H)C_\omega
%	\end{equation}
%	and for $\omega=1$
%	\begin{equation}
%	\sum\limits_{l=1}^{H-1}\sum\limits_{\mu=1}^{H-l} \Phi_{l-1}\left[
%	A^l_{H\mu} w_\mu \right] + w_1+\underbrace{\Lambda_1}_{=0} w_H \equiv 0 \quad .
%	\end{equation}
%	\begin{itemize}
%	\item If for any $H=2,3,\ldots,n$ we have $A^1_{H\mu}=0$, then $w_1\equiv 0$ and 
%	by corollary \ref{corollary:volterra_alt1:M} the system is HIO.\\
%	For example, $H=2$ shows that again $A^1_{21}=0$ is sufficient.
%	\end{itemize}
%	\begin{remark}
%	For this condition it is possible to have $n \leq p + 1$, i.e. less observables 
%	than states. 
%	\end{remark}

\clearpage
\subsubsection{$C$ has rank $p < n$}\label{subsub3}
	Following the idea of \ref{subsub2}, let $C$ be any $p\times n$ matrix and let 
	$\mathcal{I}\subset \{1,2,\ldots,n\}$ be an index set such that $\{C_i|i\in
	\mathcal{I}\}$ are linearly 
	independent and for any $H\notin \mathcal{I}$ there are unique coefficients $
	\Lambda_i^H$ such that $C_H=\sum_{i\in\mathcal{I}} \Lambda^H_i C_i$. 
	Furthermore introduce the index sets $\mathcal{H}_i$ such that $H\in \mathcal{H}_i 
	\Leftrightarrow \Lambda_i^H = 0$ and $\mathfrak{H}_i=\bigcap_{\iota\leq i}
	\mathcal{H}_\iota$. Set $i_\text{min}$ the smallest $i$ such that 
	$\mathfrak{H}_i= \emptyset $. The notation $\mathfrak{H}_{j-1}$ means the $
	\mathfrak{H}_i$ 
	with the biggest $i<j$ in $\mathcal{I}$ .
	We get
	\begin{equation}
	y(t) = \sum\limits_{i\in\mathcal{I}} \left(\varphi_i(t) + \sum\limits_{H\notin 
	\mathcal{I}}\Lambda^H_i \varphi_H(t) \right) C_i 
	\end{equation}
	setting $y\equiv 0$, equating coefficients and differentiation with respect to $t$ 
	yields for all $i\in\mathcal{I}$
	\begin{equation}
	\sum\limits_{l=1}^{i-1}\sum\limits_{\mu=1}^{i-l} \Phi_{l-1}\left[A^l_{i\mu}
	w_\mu \right]
	+ \sum\limits_{H\notin \mathcal{I}}\sum\limits_{l=1}^{H-1}\sum\limits_{\mu=1}^{H-l}
	\Lambda_i^H \Phi_{l-1}\left[A^l_{H\mu} w_\mu \right] +
	w_i + \sum\limits_{H\notin\mathcal{I}} \Lambda_i^H w_H \equiv 0\quad. 
	\label{eq:volterra_alt1:C_general}
	\end{equation}
	
	
	\paragraph*{Algorithmic Approach} If $1\in \mathcal{I}$, choose $i=1$ and 
	\eqref{eq:volterra_alt1:C_general} reduces to
	\begin{equation}
	\sum\limits_{H\notin \mathcal{I}}\sum\limits_{l=1}^{H-1}\sum\limits_{\mu=1}^{H-l}
	\Lambda_1^H \Phi_{l-1}\left[A^l_{H\mu} w_\mu \right] +
	w_1 + \sum\limits_{H\notin\mathcal{I}} \Lambda_1^H w_H \equiv 0 \quad . 
	\end{equation}
	If for all $H\notin\mathcal{I}$ we find $\Lambda^H_1 A^1_{H\mu}
	=0\forall \mu $, then $w_1\equiv 0$ and $w_H\equiv 0\forall H\notin \mathfrak{H}_1$. 
	Assume this condition holds.
	\begin{enumerate}
		\item If $2\in \mathcal{I}$, \eqref{eq:volterra_alt1:C_general} with $i=2$ yields
		\begin{equation}
		\sum\limits_{H\notin\mathcal{I}} \sum\limits_{l=1}^{H-1}
		\sum\limits_{\substack{\mu=2 \\ \mu \in\mathfrak{H}_1}}^{H-l}
		\Lambda_2^H \Phi_{l-1}\left[ A^l_{H\mu} w_\mu \right] +w_2 + 
		\sum\limits_{ H\in \mathfrak{H}_1} \Lambda_2^H w_H \equiv 0
		\end{equation}		 
		that is, if for all $H\notin \mathcal{I}$ we find 
		$\Lambda^H_2 A^1_{H\mu} = 0\forall 2\leq\mu\in \mathfrak{H}_1$, then 
		$w_2\equiv 0$ and $w_H\equiv 0 \forall H\notin \mathfrak{H}_2$.
		\item If $2\notin \mathcal{I}$ then
			\begin{enumerate}
			\item If $2\notin \mathfrak{H}_1$ then $w_2\equiv 0$.
			\item If $2\in\mathfrak{H}_1$ then 
			we have no information about $w_2$.
			\end{enumerate}
	\end{enumerate}
	If we can conclude $w_2\equiv 0$ we increase $i$ and  proceed in a similar 
	manner to get $w_1=\ldots = w_{j-1}\equiv 0$ up to an $j$ with 
	$j\in \mathfrak{H}_{j-1}$ (and consequently $j\notin\mathcal{I}$).\\
%	\begin{enumerate}
%		\item If $j+1\in\mathcal{I}$ choose $i=j+1$ and 
%		\eqref{eq:volterra_alt1:C_general} yields 
%		\begin{equation}
%		\Phi_0\left[ A^1_{(j+1)j} w_j \right] + \sum\limits_{H\notin\mathcal{I}}
%		\sum\limits_{l=0}^{H-1} \sum\limits_{\mu=j}^{H-l} \Lambda^H_{j+1} 
%		\Phi_{l-1}\left[ A^l_{H\mu} w_\mu \right] +w_{j+1} + \sum\limits_{
%		H\notin\mathcal{I}} \Lambda^H_{j+1} w_H \equiv 0 
%		\end{equation}				
%		\begin{enumerate}
%			\item If $w_j\equiv 0$ we can proceed as before.
%			\item If $A^1_{(j+1)j}=0$ we can proceed as before, but without information 
%			about $w_j$.
%			\item If $A^1_{(j+1)j} w_j\not\equiv 0$ we use the bijective property of $
%			\Phi_0$ to 
%			get
%			\begin{equation}
%			\sum\limits_{H\notin
%			\mathcal{I}}
%			\sum\limits_{l=0}^{H-1} \sum\limits_{\mu=j}^{H-l} \Lambda^H_{j+1} 
%			\Phi_{l-1}\left[ A^l_{H\mu} w_\mu \right] +w_{j+1} + \sum\limits_{
%			H\notin\mathcal{I}} \Lambda^H_{j+1} w_H \not\equiv 0
%			\end{equation}						
%			which shows, that not all $w_\mu$ can be zero and thus the system would not 
%			be HIO.
%		\end{enumerate}
%		
%		\item
%	\end{enumerate}


	Now assume there is such an $j$ and $j<n$. 
	\begin{enumerate}
		\item If $j+1 \in \mathcal{I}$ choose $i=j+1$ to get
		\begin{equation}
		\Phi_0\left[ A_{(j+1)j}^1 w_j \right]  + \sum\limits_{H\notin\mathcal{I}} \sum
		\limits_{l=1}^{H-1}
		\sum\limits_{\substack{\mu = j \\\mu\in\mathfrak{H}_j }}^{H-l} \Lambda_{j+1}^H 
		\Phi_{l-1}\left[ A^l_{H\mu} w_\mu \right]
		+w_{j+1} +
		\sum\limits_{ H\in\mathfrak{H}_j } \Lambda_{j+1}^H w_H \equiv 0 \quad .
		\end{equation} 
		To get conditions for HIO we can
		\begin{enumerate}
			\item If $A^1_{(j+1)j}=0$ proceed as before assuming $w_j\not\equiv 
			0$. In the next step we will need $A^1_{(j+2)j}=0$ etc.
			\item If $A^1_{(j+1)j}\neq0$ force $w_j \equiv 0$ and proceed as 
			before.
		\end{enumerate}
		\item If $j+1 \notin \mathcal{I}$ everything gets worse...
	\end{enumerate}

	\paragraph*{Strict Algorithm}
	\begin{enumerate}
		\item If $A$ nilpotent: Transform to strict lower triangular Matrix. Else: break.
		\item Get $\rank{C}$, choose $\mathcal{I}$, compute $\Lambda^H_i$ and 
		$\mathfrak{H}_i$.
		\item $\mathcal{N}=\emptyset$
		\item For $i=1,i\leq n$:
			 \begin{itemize}
			 \item[] If $i\in \mathcal{I}$: 
			 	\begin{itemize}
			 	\item[] If $\forall H\notin \mathcal{H}_i$ we find $\Lambda^H_i A_{H\mu}
			 	=0$ 
			 		for $\mu \in\{i,i+1,\ldots,n \}
			 		\cap \mathfrak{H}_{i-1}$:
			 		\begin{itemize}
			 			\item[] $i++$
			 		\end{itemize}
			 	\item[] Else: \begin{itemize}\item[] break \end{itemize}
			 	\end{itemize}
			 \item[] Else:
			 	\begin{itemize}
			 	\item[] If $i\in \mathfrak{H}_{i-1}$:
			 		\begin{itemize} 
			 						\item[] Force $w_i\equiv 0$ 
			 						\item[] $\mathcal{N} = \mathcal{N}\cup \{i\}$
			 						\item[] $i++$			
			 				  \end{itemize}
			 	\item[] Else: 
			 			\begin{itemize}
			 		\item[] $i++$
			 		\end{itemize}
			 	\end{itemize}
			 \end{itemize}
		\item If the iteration was successful:\\
		Under the restriction $w_i\equiv 0 \forall i\in\mathcal{N}$, the system 
		is HIO.
	\end{enumerate}
	