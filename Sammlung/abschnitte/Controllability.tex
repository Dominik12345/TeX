\section{Hidden Input Observability}
Consider the linear system $\mathcal{S}$
\begin{align}
\dot x &= Ax + Bu +Dw \label{eq:dynamics} \tag{$\mathcal{S}1$} \\
y &= Cx \label{eq:observation} \tag{$\mathcal{S}2$} \\
x(0) &= x_0 \label{eq:initial_value}\tag{$\mathcal{S}3$}
\end{align}
where $x:[0,t_\text{f}]\to \mathbb{R}^n$, $u:[0,t_\text{f}]\to \mathbb{R}^{m'}$, 
$w:[0,t_\text{f}]\to \mathbb{R}^m$ and $y:[0,t_\text{f}]\to \mathbb{R}^r$. Furthermore 
$A\in \mathbb{R}^{n\times n}$, $B\in \mathbb{R}^{n\times m'}$, $D\in\mathbb{R}^{n\times m}$ 
and $C\in\mathbb{R}^{r\times n}$.
It is well known that
\begin{equation}
y(t) = C\e^{At} x_0 + C\int\limits_0^t \e^{A(t-\tau)}\left(Bu(\tau)+Dw(\tau)\right)
\,\td \tau
\end{equation}
is a solution of $\mathcal{S}$.

\begin{definition}[Hidden Input Observability]
Let $w_1$ and $w_2$ be admissible functions and let $y_1$ and $y_2$ be the solutions 
of $\mathcal{S}$ with $w_1$ and $w_2$, respectively.
The system is called \textit{hidden input observable}, if 
\begin{equation}
w_1 \neq w_2 \quad \Rightarrow \quad y_1 \neq y_2 \quad .
\end{equation}
\end{definition}

\begin{theorem}[Target Controllability] \label{theorem:target_controllability}
For the system $\mathcal{S}$ with $B=0$ define the matrix
\begin{equation}
M_k = [CD,CAD,CA^2D,\ldots,CA^{k-1}D] \quad .
\end{equation}
By the Cayley-Hamilton theorem we know that this system is target controllable, i.e. any 
$y_\text{f}\in\mathbb{R}^r$ can be reached within a finite time interval, if and only if 
\begin{equation}
\rank{M_n} = r \quad . \label{eq:rank_condition}
\end{equation}
\end{theorem}

\begin{remark}
Since $M_n$ is a $r \times nm$ matrix, the rank condition \eqref{eq:rank_condition} needs 
$r \leq nm$. Obviously only $m \leq n$ makes sense and hence $r \leq n^2$.
\end{remark}


We now search for a sufficient or necessary condition for hidden input observability. 
Let $y_1$ and $y_2$ be solutions of $\mathcal{S}$ with $w_1$ and $w_2$, respectively. If 
we define $y=y_1-y_2$ and $w=w_1-w_2$, then
\begin{equation}
y(t) = C \int\limits_0^t \e^{A(t-\tau)}D w(\tau) \,\td \tau \label{eq:solution_S'}
\end{equation}
is a solution of $\mathcal{S}$ with $B=0$ and $x_0=0$. We call this simplified system 
$\mathcal{S}'$. Furthermore $w_1 \neq w_2 
\Leftrightarrow w\not\equiv 0$ and for $y$ analogous.

\subsection{necessary condition}
Assume the system is hidden input observable, i.e. $w\not\equiv 0 \Rightarrow 
y\not\equiv 0$. 
\begin{lemma}
If $\mathcal{S}$ is hidden input observable, then 
\begin{align}
&w\not\equiv 0 \Leftrightarrow y\not\equiv 0 \\
\text{or equivalently} \quad & w \equiv 0 \Leftrightarrow y\equiv 0 \quad .
\end{align}
\end{lemma}
\begin{proof}
\begin{enumerate}
\item $w\not\equiv 0 \Rightarrow y\not\equiv 0$ and equivalently $y\equiv 0 \Rightarrow 
w\equiv 0$ by definition of hidden input 
observability and $w\equiv 0 \Rightarrow y\equiv 0$ is obvious.
\item Assume $y\not\equiv 0$ and $w \equiv 0$. The explicit formula 
\eqref{eq:solution_S'} results in $y\equiv 0$ which is in conflict with the assumption. 
Thus $y\not\equiv 0 \Rightarrow w \not\equiv 0$.
\end{enumerate} 
\end{proof}
Expanding \eqref{eq:solution_S'} in a power series we get
\begin{equation}
y(t) = \sum\limits_{k=0}^\infty CA^kD \int\limits_0^t\frac{(t-\tau)^kw(\tau)}{k!} \td 
\tau \quad ,
\end{equation}
%Since $w$ could be any function in $\mathbb{R}^m$ and because the monomials are linearly 
%independent we can see that
%\begin{equation}
%y \equiv 0 \quad \text{if and only if} \quad C A^k D v_k = 0 \,\forall v_k 
%\in \mathbb{R}^m \,\text{and}\, k=0,1,\ldots 
%\end{equation}
%and thus
%\begin{equation}
%y\not\equiv 0 \quad \Rightarrow \quad \exists k \, | \, \kernel{CA^kD} = \O
%\end{equation}
%that is $\rank{CA^kD}=m$. Each $CA^kD$ is a $r\times m$ matrix, so this is only 
%possible if $r \geq m$.
and by the Cayley-Hamilton theorem we know that 
this is equivalent to
\begin{equation}
y(t) = M_n V(t)
\end{equation}
with an arbitrary $V:[0,t_\text{f}]\to \mathbb{R}^{nm}$. 
\begin{lemma}
We have
\begin{equation}
\rank{M_n} = nm .
\end{equation}
\begin{proof}
\begin{enumerate}
\item Let $V(t) \in \kernel{M_n} \,\forall t$. Then $y \equiv 0 \Rightarrow w\equiv 0 
\Rightarrow V \equiv 0$.
\item Now $V \equiv 0$ then $V(t)\in \kernel{M_n}$ because $M_n$ is linear.
\end{enumerate}
Hence $\dim{\kernel{M}} = 0$. By the rank theorem 
\begin{equation}
f:X\to Y \,\text{linear} \quad \text{then} \quad \dim{V} = \dim{\kernel{f}} + \rank{f}
\end{equation}
we see
\begin{equation}
M_n:\mathbb{R}^{nm}\to\mathbb{R}^r \tab{and} nm = \rank{M_n} \quad . 
\end{equation}
\end{proof}
\end{lemma}
\begin{remark}
The preceding lemma only makes sense if $r \geq nm$ because $\rank{M_n}\leq \min(nm,r)$. 
The theorem \ref{theorem:target_controllability} needs $r \leq nm$, so that we can 
conclude: \\

A system $\mathcal{S}$ can be hidden input observable only if $r\geq nm$. \\

A system $\mathcal{S}'$ that is target controllable can belong to a system $\mathcal{S}$ 
that is hidden input observable only if $r = nm$.
\end{remark}


\subsection{sufficient condition}
