\section{Monomials}\label{section:Monom}
%Is it possible to chose a $w:[0,t]\to \mathbb{R}^m$ in such a way, that 
%\begin{equation}
%v_l := \int\limits_0^t \frac{(t-\tau)^2}{l !} w(\tau) \td \tau \quad l =
%0,1,\ldots ,n
%\end{equation}
%can be a arbitrary set of vectors in $\mathbb{R}^m$?
%\\
%So see that this is possible we do a shift in variables $T:=t-\tau$ and $W(T):= w(t-T)/l!$ 
%to get
%\begin{equation}
%v_l = \int\limits_0^t T^l W(T) \td T \quad .
%\end{equation}
%This is equivalent to 
%\begin{equation}
%\int\limits_0^t \begin{pmatrix}
%W_1(T) & T W_1(T) & \hdots & T^n W_1(T) \\
%W_2(T) & T W_2(T) & \hdots & T^n W_2(T) \\
%\vdots & \vdots   &  \ddots      &  \vdots    \\
%W_m(T) & T W_m(T) & \hdots & T^n W_m(T)
%\end{pmatrix}
%\td T = V
%\end{equation}
%where $V_{ij}$ is the $i$-th component of $v_j$. We can make the Ansatz 
%\begin{equation}
%W_i(T) = \sum\limits_{k=0}^F X^i_k T^k 
%\end{equation}
%then
%\begin{equation}
%V_{ij} = \int\limits_0^t T^j W_i(T) \td T 
%= \int\limits_0^t \sum\limits_{k=0}^F X^i_k T^{k+j} \td T 
%= \sum\limits_{k=0}^F X^i_k \frac{1}{k+j+1} t^{k+j+1}
%\end{equation}
A compilation of integral lemmas, not rigorously proved.
\begin{lemma}
Let $w:[0,t_\text{f}]\to \mathbb{R}^m$ be Riemann integrable. 
\begin{equation}
\int\limits_0^t w(\tau)\,\td \tau =0 \quad \forall t\in [0,t_\text{f}] \quad \Rightarrow 
\quad w \equiv 0
\end{equation}
\begin{proof}
Define $\Delta t = t_\text{f}/N$ sufficient small and $N_t$ such that $N_t \Delta t = t$. 
Then 
\begin{equation}
\lim_{N\to \infty} \Delta t \sum\limits_{k=0}^{N_t}  w(k \Delta t) = 0 \quad \forall N_t
\end{equation}
which means
\begin{align}
w(0) &= 0 \\
w(0)+w(\Delta t) = w(\Delta t) &= 0\\
w(0)+w(\Delta t) + w(2\Delta t)= w(2\Delta t) &= 0 \\
\vdots   
\end{align}
\end{proof}
\end{lemma}
\begin{corollary}
Assume there is a function $w$ such that
\begin{equation}
\int\limits_0^t w(\tau) \,\td  \tau = \int\limits_0^t \tau w(\tau) \,\td \tau \quad 
\forall t
\end{equation}
which is equivalent to
\begin{equation}
\int\limits_0^t w(\tau)(1-\tau) \, \td \tau = 0 \quad \forall t \quad .
\end{equation}
By the preceding lemma we know $w(\tau)(1-\tau)= 0 \, \forall \tau$ and hence 
$w(t)=0\,\text{a.e.}$ and because $w$ is Riemann integrable $w\equiv 0$. With 
$\tilde{w}(\tau) = \tau w(\tau)$ we get $\tau w(\tau)(1-\tau)=0\,\forall \tau$. Thus 
if for any $N$ and for all $t$
\begin{equation}
\int\limits_0^t \tau^N w(\tau) \,\td\tau= 0
\end{equation}
then $w\equiv 0$.
\end{corollary}