% This is a most basic header

% article with standard layout
\documentclass[12pt,a4paper]{article}

%encoding text
\usepackage[utf8]{inputenc}
\usepackage[english]{babel}

%bibliography package
\usepackage{csquotes}
\usepackage[backend = biber]{biblatex} 


%fond
\usepackage[T1]{fontenc}
\usepackage{upgreek}

% math formulas
\usepackage{amsmath}	% standard math equations
%\usepackage[fleqn]{amsmath} % left aligned equations
\usepackage{amsfonts}
\usepackage{amssymb}
\usepackage{amsthm}		% theorem environments
\usepackage{nicefrac}	% make nicefracs ( 1/2 with smaller slash )

% math fond
\usepackage{fourier}

% fancy layout
\usepackage{tcolorbox}

%graphics
\usepackage{graphicx}
\usepackage{wrapfig}
\usepackage{pgfplots} % to simply input matplotlib results

%tables
\usepackage{booktabs}

%layout
\setlength{\parindent}{0pt}


\newcommand{\td}{\ensuremath{\text{d}} }


\newcommand{\tab}[1]{\ensuremath{\quad \text{#1} \quad}}
\newcommand{\e}{\ensuremath{\text{e}}}

\newcommand{\adj}[1]{\ensuremath{ \, \text{adj}\,#1 \,\, }}
\renewcommand{\det}[1]{\ensuremath{ \, \text{det}\,#1 \,\, }}
\newcommand{\rank}[1]{\ensuremath{ \, \text{rank}\,#1 \,\, }}
\newcommand{\kernel}[1]{\ensuremath{ \, \text{kernel}\,#1 \,\,}}
\renewcommand{\dim}[1]{\ensuremath{ \, \text{dim}\,#1 \,\,}}
\newcommand{\linspan}[1]{\ensuremath{ \, \text{span}\,#1 \,\,}}
\newcommand{\supremum}[2]{\ensuremath{ \, \underset{#2}{\text{sup}}\,#1 \,\,}}
\theoremstyle{definition}
\newtheorem{definition}{Definition}
\newtheorem{theorem}{Theorem}
\newtheorem{proposition}{Proposition}
\newtheorem*{corollary}{Corollary}
\newtheorem{example}{Example}
\newtheorem*{remark}{Remark}
\newtheorem{lemma}{Lemma}
%\newtheorem*{proof}{Proof}

\begin{document}

Minimiere 
\begin{equation}
J = \int\limits_0^T \sum\limits_{k=1}^p |y(t) - h(x(t))|^2 \, \td t + \frac{\alpha_2}{2}
\int\limits_0^T \sum\limits_{k=1}^n D_{kk} |w_k(t)|^2 \, \td t
\end{equation}
unter
\begin{equation}
\dot{x}(t) = \tilde{f}(x(t)) + D w(t) \quad .
\end{equation}
Dabei ist $D$ eine Diagonalmatrix aus $0$ und $1$ (um bestimmte $w$s zu unterdrücken).
Die Hamiltonfunktion lautet
\begin{equation}
H(x,w,\lambda,t) = \sum\limits_{k=1}^p |y(t) - h(x(t))|^2 + \frac{\alpha_2}{2}\sum\limits_{k=1}^n D_{kk} |w_k(t)|^2 + \sum\limits_{k=1}^n \lambda_k (\tilde{f}_k(x(t) + D_{kk} w_k(t))
\end{equation}
mit den kanonischen Gleichungen
\begin{align}
&\left. 
\begin{aligned}
\frac{\partial H}{\partial x_i} &= -\dot{\lambda}_i \\
\frac{\partial H}{\partial \lambda_i} &= \dot{x}_i 
\end{aligned}
\right\}
\text{Hamilton Gleichungen} \\
&\frac{\partial H}{\partial w_i} = 0 \quad \text{Extremale Lösung}  \quad .
\label{eq:2}
\end{align}
Gegeben seien Anfangswerte $x_i(0)=x_{0,i}$.
Falls $x_i(T)$ frei gelassen werden, folgt $\lambda_i(T)=0$ und aus \eqref{eq:2} für alle 
$i$ mit $D_{ii}= 1$
\begin{equation}
\alpha_2 w_i(t) + \lambda_i(t) = 0 \quad \forall t\in[0,T]
\end{equation}
und daher $w_i(T)=0$.
Werden aber $x_i(T)$ vorgegeben, dann gibt es keine Bedingung an $\lambda_i(T)$. Problem: 
$x_i(T)$ sind unbekannt bzw. müssen erst ermittelt werden. Die Endwerte $\tilde{x}_i(T)$ 
des nominalen Modells sind im ersten Iterationsschritt sinnvoll, danach müssten die 
Endwerte aber mit angepasst werden. Möglicherweise (weiß ich aber nicht sicher) würden 
sich die Randwerte $x_i(T)$ aber auch beim Iterieren nicht ändern, weil der Startwert 
dann wieder festgepinnt ist. Alternativ kann man (mit zwei zugedrückten Augen) folgende 
Rechnung anstellen (vektoriell geschrieben, $\cong$ ist Gleicheit für $\epsilon\to0$):\\
(Du kannst auch direkt zum Ergebnis springen)\\


Sei $h^\dagger$ die Inverse (falls existent) bzw. Pseudoinverse (wie man die definiert muss ich mir noch überlegen) von $h$. Die Messwerte sind $y(t)$. Als Endbedingung nehmen wir $h(x(T)) = y(T)$ bzw. $x(T)=h^\dagger(y(T))$ (ähnlich wie im DEN paper). Sei $
\epsilon>0$ klein, möglichst so, dass $y(T-\epsilon)$ ein Messwert ist.
\begin{equation}
x(T-\epsilon) \cong x(T) - \dot{x}(T) \epsilon = h^\dagger(y(T))-\epsilon 
\left[ \tilde{f}(h^\dagger(y(T)))-\frac{1}{\alpha_2} \lambda(T) \right]
\end{equation}
außerdem
\begin{equation}
h^\dagger( y(T-\epsilon) ) \cong h^\dagger (y(T)) - \td h^\dagger_{y(T)} \dot{y}(T)
\epsilon
\end{equation}
dabei ist (glaube ich) $\td h^\dagger_{y(T)}$ die Jacobimatrix bzw. das Differenzial von 
$h^\dagger$ ausgewertet an $y(T)$.
Zusammenbauen und nach $\lambda$ umstellen liefert
\begin{equation}
\lambda(T) = \alpha_2 \left\{
\tilde{f}\left(h^\dagger(y(T)) \right) - \td h^\dagger_{y(T)} \dot{y}(T)
 \right\} \quad .
\end{equation}
Wenn man annimmt, dass $y(t)$ prinzipiell differenzierbar ist, kann man $\dot{y}(T)$ 
numerisch ermittelt, die Inverse bzw. Pseudoinverse sollte kein größeres Problem sein, 
da man für kleine $\epsilon$ zur Not noch $h$ linearisieren kann und dann die 
Moore-Penrose Inverse nimmt.




\end{document}