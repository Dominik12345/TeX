\section{Methods}
The \textsf{seeds} package provides two distinct ways to compute systematic errors. Both make 
use of the Hamilton formalism, a convenient tool in dynamic optimization. 
As a setup for the DEN you must provide
\begin{enumerate}
	\item a mathematical model $f$ of the dynamic system,
	\item an observation function $h$,
	\item measured data $y^\text{obs}$. 
\end{enumerate}
The \textsf{greedyseeds} method is a deterministic algorithm based on a dynamic version of the 
method of steepest descent, combined with a "greedy" selection process to archive a sparse 
solution. In addition to the systems equations it is possible to constrain the system to the 
solution space of algebraic equations, e.g. mass conservation. In particular the case
\begin{equation}
	\frac{\partial f_i}{\partial x_i} = 0 \quad ,
\end{equation} 
where the index is understood as the $i$-th component, may lead to certain numerical 
difficulties and highly benefits of such an algebraic constraint.
\\

The \textsf{bayesianseeds} method is a stochastic algorithm based on ideas of bayesian 
inference and uses LASSO regularization to generate sparsity of the solution \cite{BDEN}. 
In addition to 
a specific estimate of the unknown inputs it results in a probability density in the 
space of unknown input functions.